\documentclass[a4paper,12pt,twoside]{report}
\usepackage[utf8]{inputenc}
\usepackage[T1]{fontenc}
\usepackage[french]{babel}
\usepackage{geometry}
\usepackage{graphicx}
\usepackage{hyperref}
\usepackage{listings}
\usepackage{xcolor}
\usepackage{tikz}
\usepackage{float}
\usepackage{fancyhdr}
\usepackage{tabularx}
\usepackage{booktabs}
\usepackage{enumitem}
\usepackage{amsmath}
\usepackage{amssymb}

% Configuration des marges
\geometry{
    top=2.5cm,
    bottom=2.5cm,
    left=2.5cm,
    right=2.5cm,
    headheight=15pt
}

% Configuration des headers/footers
\pagestyle{fancy}
\fancyhf{}
\fancyhead[LE,RO]{\thepage}
\fancyhead[LO]{\nouppercase{\leftmark}}
\fancyhead[RE]{\nouppercase{Rapport Technique - Synf Project}}
\renewcommand{\headrulewidth}{0.4pt}
\renewcommand{\footrulewidth}{0pt}

% Configuration des liens hypertextes
\hypersetup{
    colorlinks=true,
    linkcolor=blue!70!black,
    filecolor=magenta,      
    urlcolor=blue!60!black,
    citecolor=green!50!black,
    pdftitle={Rapport Technique Complet - Synf Project E-commerce Symfony 7},
    pdfauthor={Equipe de Developpement},
    pdfsubject={Documentation Technique Complete},
    pdfkeywords={Symfony, PHP, E-commerce, Doctrine, OAuth, Securite}
}

% Couleurs personnalisees
\definecolor{codegreen}{rgb}{0,0.6,0}
\definecolor{codegray}{rgb}{0.5,0.5,0.5}
\definecolor{codepurple}{rgb}{0.58,0,0.82}
\definecolor{backcolour}{rgb}{0.97,0.97,0.95}
\definecolor{symfonycolor}{RGB}{0, 0, 0}
\definecolor{phpcolor}{RGB}{119, 123, 180}
\definecolor{doctrinecolor}{RGB}{252, 108, 38}
\definecolor{twigcolor}{RGB}{186, 205, 34}

% Configuration pour le code source PHP
\lstdefinestyle{phpstyle}{
    language=PHP,
    backgroundcolor=\color{backcolour},   
    commentstyle=\color{codegreen}\itshape,
    keywordstyle=\color{phpcolor}\bfseries,
    numberstyle=\tiny\color{codegray},
    stringstyle=\color{codepurple},
    basicstyle=\ttfamily\footnotesize,
    breakatwhitespace=false,         
    breaklines=true,                 
    captionpos=b,                    
    keepspaces=true,                 
    numbers=left,                    
    numbersep=5pt,                  
    showspaces=false,                
    showstringspaces=false,
    showtabs=false,                  
    tabsize=2,
    frame=single,
    rulecolor=\color{codegray},
    morekeywords={class, function, public, private, protected, static, return, new, use, namespace, extends, implements, readonly, array, Collection, self, mixed},
    emph={Entity, Repository, Controller, Service, Form},
    emphstyle=\color{doctrinecolor}\bfseries
}

% Configuration pour YAML
\lstdefinestyle{yamlstyle}{
    backgroundcolor=\color{backcolour},
    basicstyle=\ttfamily\footnotesize,
    breaklines=true,
    numbers=left,
    numbersep=5pt,
    frame=single,
    rulecolor=\color{codegray},
    keywordstyle=\color{blue!70!black}\bfseries,
    commentstyle=\color{codegreen}\itshape,
    stringstyle=\color{codepurple}
}

% Configuration pour Twig
\lstdefinestyle{twigstyle}{
    backgroundcolor=\color{backcolour},
    basicstyle=\ttfamily\footnotesize,
    breaklines=true,
    numbers=left,
    numbersep=5pt,
    frame=single,
    rulecolor=\color{codegray},
    keywordstyle=\color{twigcolor}\bfseries,
    commentstyle=\color{codegreen}\itshape,
    stringstyle=\color{codepurple},
    morekeywords={block, endblock, extends, if, endif, for, endfor, include}
}

\lstset{style=phpstyle}

% TikZ libraries
\usetikzlibrary{shapes.geometric,arrows.meta,positioning,fit,backgrounds,shadows,calc,decorations.pathreplacing}

% Custom commands
\newcommand{\code}[1]{\texttt{\textcolor{codepurple}{#1}}}
\newcommand{\entity}[1]{\texttt{\textcolor{doctrinecolor}{\bfseries #1}}}
\newcommand{\controller}[1]{\texttt{\textcolor{blue!70!black}{\bfseries #1}}}
\newcommand{\service}[1]{\texttt{\textcolor{codegreen}{\bfseries #1}}}

\begin{document}

% ============================================================================
%                           PAGE DE GARDE
% ============================================================================
\begin{titlepage}
    \begin{center}
        \vspace*{1cm}
        
        \begin{tikzpicture}
            \node[draw=blue!60!black, line width=3pt, rounded corners=15pt, inner sep=20pt, fill=blue!5] {
                \begin{minipage}{0.8\textwidth}
                    \centering
                    \Huge\bfseries\textcolor{blue!70!black}{Rapport Technique Detaille}
                    
                    \vspace{0.5cm}
                    \Large\textcolor{gray}{Projet FreshMarket E-commerce}
                \end{minipage}
            };
        \end{tikzpicture}
        
        \vspace{1.5cm}
        
        \begin{tikzpicture}
            \node[circle, draw=blue!60!black, line width=2pt, minimum size=3cm, fill=blue!10] (logo) {};
            \node at (logo) {\Huge $\blacksquare$};
        \end{tikzpicture}
        
        \vspace{1.5cm}
        
        \Large
        \begin{tabular}{rl}
            \textbf{Framework :} & Symfony 7.3 \\
            \textbf{Langage :} & PHP 8.x \\
            \textbf{Base de Donnees :} & MySQL / MariaDB \\
            \textbf{ORM :} & Doctrine 3.x \\
            \textbf{Template :} & Twig 3.x \\
        \end{tabular}
        
        \vfill
        
        \begin{tikzpicture}
            \node[draw=gray, rounded corners=5pt, inner sep=15pt, fill=gray!5] {
                \begin{minipage}{0.6\textwidth}
                    \centering
                    \large
                    \textbf{Auteur :} Equipe de Developpement\\[0.3cm]
                    \textbf{Version :} 1.0\\[0.3cm]
                    \textbf{Date :} \today
                \end{minipage}
            };
        \end{tikzpicture}
        
    \end{center}
\end{titlepage}

% ============================================================================
%                           TABLE DES MATIERES
% ============================================================================
\tableofcontents
\newpage

% ============================================================================
%                           CHAPITRE 1 : INTRODUCTION
% ============================================================================
\chapter{Introduction Generale}

\section{Presentation du Projet}

Le projet \textbf{FreshMarket} (aussi connu sous le nom de code \texttt{synf\_project}) est une application web complete de type \textbf{e-commerce} et \textbf{systeme de reservation} developpee avec le framework \textbf{Symfony 7.3}. Cette plateforme permet aux utilisateurs de :

\begin{itemize}[leftmargin=2cm]
    \item S'inscrire et gerer leur profil utilisateur
    \item Se connecter via OAuth (Google, Facebook)
    \item Consulter un catalogue de produits avec gestion du stock
    \item Ajouter des produits au panier et passer des commandes
    \item Effectuer des reservations de services
    \item Consulter des statistiques personnalisees
\end{itemize}

\section{Objectif de ce Rapport}

Ce document technique a pour objectif de fournir une \textbf{documentation exhaustive} permettant a tout developpeur de comprendre \textbf{100\% du projet techniquement}. Chaque fichier, chaque fonction, chaque relation est expliquee en detail avec des exemples de code commentes.

\section{Technologies Utilisees - Vue d'Ensemble}

\begin{table}[H]
\centering
\caption{Stack Technologique Complete}
\begin{tabularx}{\textwidth}{|l|X|l|}
\hline
\textbf{Categorie} & \textbf{Technologie} & \textbf{Version} \\
\hline
\hline
Framework & Symfony & 7.3 \\
\hline
Langage & PHP & 8.x \\
\hline
ORM & Doctrine (DBAL + ORM) & 3.x \\
\hline
Template Engine & Twig & 3.x \\
\hline
Authentification & Symfony Security + OAuth2 & N/A \\
\hline
OAuth Library & knpuniversity/oauth2-client-bundle & 2.19 \\
\hline
Frontend JS & Stimulus (Symfony UX) & 2.30 \\
\hline
SPA Enhancement & Symfony UX Turbo & 2.30 \\
\hline
CSS Framework & Bootstrap 5 & 5.3 \\
\hline
Validation & Symfony Validator & 7.3 \\
\hline
Mailing & Symfony Mailer & 7.4 \\
\hline
Asset Management & Symfony AssetMapper & 7.3 \\
\hline
\end{tabularx}
\end{table}

% ============================================================================
%                           CHAPITRE 2 : ARCHITECTURE
% ============================================================================
\chapter{Architecture du Projet}

\section{Structure MVC de Symfony}

Symfony impose une architecture \textbf{MVC} (Model-View-Controller) stricte. Le projet respecte cette separation :

\begin{figure}[H]
\centering
\begin{tikzpicture}[
    node distance=2cm,
    box/.style={rectangle, draw=blue!60!black, line width=1.5pt, rounded corners=8pt, minimum width=4cm, minimum height=1.2cm, text centered, fill=blue!10, font=\bfseries},
    arrow/.style={-{Stealth[length=3mm]}, thick, blue!60!black}
]
    % Nodes
    \node[box] (browser) {Navigateur (Client)};
    \node[box, below=of browser] (controller) {Controller};
    \node[box, left=3cm of controller] (model) {Model (Entity)};
    \node[box, right=3cm of controller] (view) {View (Twig)};
    \node[box, below=of model] (repository) {Repository};
    \node[box, below=of repository] (database) {Base de Donnees};
    
    % Arrows
    \draw[arrow] (browser) -- node[right] {HTTP Request} (controller);
    \draw[arrow] (controller) -- node[above] {Donnees} (model);
    \draw[arrow] (model) -- (repository);
    \draw[arrow] (repository) -- node[left] {SQL} (database);
    \draw[arrow] (database) -- (repository);
    \draw[arrow] (repository) -- (model);
    \draw[arrow] (model) -- (controller);
    \draw[arrow] (controller) -- node[above] {Variables} (view);
    \draw[arrow] (view) -- (browser);
    
\end{tikzpicture}
\caption{Flux MVC dans Symfony}
\end{figure}

\section{Arborescence des Fichiers}

\begin{lstlisting}[style=yamlstyle, caption={Structure du projet}]
synf_project/
|-- src/
|   |-- Controller/        # Logique HTTP (routes)
|   |   |-- CartController.php
|   |   |-- HomeController.php
|   |   |-- ProductController.php
|   |   |-- ProfileController.php
|   |   |-- SecurityController.php
|   |   |-- GoogleOAuthController.php
|   |   |-- FacebookOAuthController.php
|   |   |-- Admin/         # Controleurs administration
|   |-- Entity/            # Modeles Doctrine (tables BDD)
|   |   |-- User.php
|   |   |-- Product.php
|   |   |-- Order.php
|   |   |-- OrderItem.php
|   |   |-- Reservation.php
|   |-- Repository/        # Requetes SQL personnalisees
|   |-- Service/           # Logique metier reutilisable
|   |   |-- CartService.php
|   |   |-- FileUploader.php
|   |-- Form/              # Formulaires Symfony
|   |-- Security/          # Authenticators OAuth
|   |-- EventSubscriber/   # Listeners d'evenements
|-- templates/             # Fichiers Twig (Vue)
|-- config/                # Configuration YAML
|-- public/                # Point d'entree web
|-- vendor/                # Dependances Composer
\end{lstlisting}

% ============================================================================
%                    CHAPITRE 3 : ENTITES DOCTRINE (MODELE)
% ============================================================================
\chapter{Entites Doctrine - Le Modele de Donnees}

\section{Vue d'Ensemble des Entites}

Le modele de donnees est compose de 5 entites principales interconnectees :

\begin{figure}[H]
\centering
\begin{tikzpicture}[
    entity/.style={
        rectangle, draw=orange!80!black, line width=2pt, 
        rounded corners=5pt, minimum width=3.5cm, minimum height=1cm,
        fill=orange!10, font=\bfseries\ttfamily
    },
    relation/.style={-{Stealth}, thick, gray}
]
    \node[entity] (user) at (0,0) {User};
    \node[entity] (order) at (-4,-3) {Order};
    \node[entity] (reservation) at (4,-3) {Reservation};
    \node[entity] (orderitem) at (-4,-6) {OrderItem};
    \node[entity] (product) at (0,-6) {Product};
    
    \draw[relation] (user) -- node[left, font=\small] {1..*} (order);
    \draw[relation] (user) -- node[right, font=\small] {1..*} (reservation);
    \draw[relation] (order) -- node[left, font=\small] {1..*} (orderitem);
    \draw[relation] (orderitem) -- node[above, font=\small] {*.1} (product);
\end{tikzpicture}
\caption{Diagramme des Relations entre Entites}
\end{figure}

\section{Entite \texttt{Product} - Etude de Cas Detaillee}

L'entite \entity{Product} represente un article du catalogue. Voici une analyse \textbf{ligne par ligne} de son code source :

\subsection{Declaration et Attributs ORM}

\begin{lstlisting}[style=phpstyle, caption={Product.php - Declaration}]
<?php

namespace App\Entity;

use App\Repository\ProductRepository;
use Doctrine\Common\Collections\ArrayCollection;
use Doctrine\Common\Collections\Collection;
use Doctrine\DBAL\Types\Types;
use Doctrine\ORM\Mapping as ORM;
use Symfony\Component\Validator\Constraints as Assert;

#[ORM\Entity(repositoryClass: ProductRepository::class)]
class Product
{
\end{lstlisting}

\textbf{Explications :}
\begin{itemize}
    \item \textbf{Ligne 3-10 :} Imports des classes necessaires. \code{ORM} contient les annotations Doctrine, \code{Assert} les contraintes de validation.
    \item \textbf{Ligne 12 :} L'attribut PHP 8 \code{\#[ORM\textbackslash Entity]} indique a Doctrine que cette classe est une entite persistee en base. Le parametre \code{repositoryClass} lie cette entite a son Repository personnalise.
\end{itemize}

\subsection{Proprietes et Colonnes}

\begin{lstlisting}[style=phpstyle, caption={Product.php - Proprietes}]
    #[ORM\Id]
    #[ORM\GeneratedValue]
    #[ORM\Column]
    private ?int $id = null;

    #[ORM\Column(length: 255)]
    #[Assert\NotBlank(message: 'Le nom est obligatoire')]
    private ?string $nom = null;

    #[ORM\Column(type: Types::DECIMAL, precision: 10, scale: 2)]
    #[Assert\NotBlank(message: 'Le prix est obligatoire')]
    #[Assert\Positive(message: 'Le prix doit etre positif')]
    private ?string $prix = null;

    #[ORM\Column]
    private ?int $stock = 0;

    #[ORM\Column]
    private ?bool $disponible = true;
\end{lstlisting}

\begin{table}[H]
\centering
\caption{Mapping des Proprietes vers la Base de Donnees}
\begin{tabularx}{\textwidth}{|l|l|X|}
\hline
\textbf{Propriete PHP} & \textbf{Type SQL} & \textbf{Description} \\
\hline
\hline
\code{\$id} & INT AUTO\_INCREMENT & Cle primaire generee automatiquement \\
\hline
\code{\$nom} & VARCHAR(255) & Nom du produit, ne peut pas etre vide \\
\hline
\code{\$prix} & DECIMAL(10,2) & Prix en euros avec 2 decimales \\
\hline
\code{\$stock} & INT & Quantite disponible en stock \\
\hline
\code{\$disponible} & TINYINT(1) & Booleen indiquant si le produit est actif \\
\hline
\end{tabularx}
\end{table}

\subsection{Relations avec OrderItem}

\begin{lstlisting}[style=phpstyle, caption={Product.php - Relation OneToMany}]
    #[ORM\OneToMany(mappedBy: 'product', targetEntity: OrderItem::class)]
    private Collection $orderItems;

    public function __construct()
    {
        $this->createdAt = new \DateTime();
        $this->orderItems = new ArrayCollection();
    }
\end{lstlisting}

\textbf{Explication de la relation :}
\begin{itemize}
    \item \code{OneToMany} : Un produit peut apparaitre dans plusieurs lignes de commande.
    \item \code{mappedBy: 'product'} : La cle etrangere est definie du cote \entity{OrderItem} dans sa propriete \code{\$product}.
    \item \code{ArrayCollection} : Collection Doctrine optimisee pour le lazy loading.
\end{itemize}

\subsection{Getters et Setters - Pattern Fluent}

\begin{lstlisting}[style=phpstyle, caption={Product.php - Methodes Fluent}]
    public function getNom(): ?string
    {
        return $this->nom;
    }

    public function setNom(string $nom): static
    {
        $this->nom = $nom;
        return $this;  // Pattern Fluent Interface
    }
\end{lstlisting}

Le pattern \textbf{Fluent Interface} permet d'enchainer les appels :
\begin{lstlisting}[style=phpstyle]
$product = (new Product())
    ->setNom('Pommes Bio')
    ->setPrix('2.99')
    ->setStock(100)
    ->setDisponible(true);
\end{lstlisting}

\section{Entite \texttt{User} - Authentification}

L'entite \entity{User} implemente les interfaces de securite Symfony :

\begin{lstlisting}[style=phpstyle, caption={User.php - Interfaces de Securite}]
class User implements UserInterface, PasswordAuthenticatedUserInterface
{
    #[ORM\Column(length: 180, unique: true)]
    private ?string $email = null;

    #[ORM\Column]
    private array $roles = [];

    #[ORM\Column]
    private ?string $password = null;

    #[ORM\Column(length: 255, nullable: true, unique: true)]
    private ?string $googleId = null;

    #[ORM\Column(length: 255, nullable: true, unique: true)]
    private ?string $facebookId = null;
\end{lstlisting}

\textbf{Points cles :}
\begin{description}
    \item[\code{UserInterface}] Requiert les methodes \code{getUserIdentifier()}, \code{getRoles()}, \code{eraseCredentials()}.
    \item[\code{PasswordAuthenticatedUserInterface}] Requiert \code{getPassword()}.
    \item[\code{googleId} / \code{facebookId}] Permettent l'association OAuth sans doublon.
\end{description}

\section{Diagramme de Classes UML Complet}

\begin{figure}[H]
\centering
\begin{tikzpicture}[scale=0.8, transform shape,
    classbox/.style={
        rectangle, draw=black, line width=1pt,
        minimum width=4.5cm, minimum height=3cm,
        fill=blue!5, font=\small\ttfamily,
        align=center
    },
    arrow/.style={-{Stealth}, thick}
]

    % User Class
    \node[classbox] (User) at (0,0) {
        \textbf{User}\\[3pt]
        \rule{4cm}{0.4pt}\\
        - id : int\\
        - email : string\\
        - password : string\\
        - roles : array\\
        - googleId : string\\
        - facebookId : string\\
        \rule{4cm}{0.4pt}\\
        + getRoles()\\
        + getFullName()
    };

    % Product Class
    \node[classbox] (Product) at (8,0) {
        \textbf{Product}\\[3pt]
        \rule{4cm}{0.4pt}\\
        - id : int\\
        - nom : string\\
        - prix : decimal\\
        - stock : int\\
        - disponible : bool\\
        \rule{4cm}{0.4pt}\\
        + getNom()\\
        + getPrix()
    };

    % Order Class
    \node[classbox] (Order) at (0,-6) {
        \textbf{Order}\\[3pt]
        \rule{4cm}{0.4pt}\\
        - id : int\\
        - numeroCommande : string\\
        - montantTotal : decimal\\
        - statut : string\\
        \rule{4cm}{0.4pt}\\
        + calculateTotal()\\
        + getStatutLabel()
    };

    % OrderItem Class
    \node[classbox] (OrderItem) at (8,-6) {
        \textbf{OrderItem}\\[3pt]
        \rule{4cm}{0.4pt}\\
        - id : int\\
        - quantite : int\\
        - prixUnitaire : decimal\\
        \rule{4cm}{0.4pt}\\
        + getTotal()
    };

    % Reservation Class
    \node[classbox] (Reservation) at (4,-11) {
        \textbf{Reservation}\\[3pt]
        \rule{4cm}{0.4pt}\\
        - id : int\\
        - dateReservation : date\\
        - nombrePersonnes : int\\
        - statut : string\\
        \rule{4cm}{0.4pt}\\
        + getStatutLabel()
    };

    % Relations
    \draw[arrow] (Order) -- node[left] {user} (User);
    \draw[arrow] (Reservation) -- node[right] {user} (User);
    \draw[arrow] (OrderItem) -- node[above] {commande} (Order);
    \draw[arrow] (OrderItem) -- node[right] {product} (Product);

\end{tikzpicture}
\caption{Diagramme de Classes UML Complet}
\end{figure}

% ============================================================================
%                    CHAPITRE 4 : CONTROLEURS
% ============================================================================
\chapter{Controleurs - La Logique HTTP}

\section{Role des Controleurs dans Symfony}

Un controleur est une classe PHP dont les methodes (appelees \textbf{actions}) repondent aux requetes HTTP. Chaque action :
\begin{enumerate}
    \item Recoit un objet \code{Request}
    \item Traite la logique metier (souvent via des Services)
    \item Retourne un objet \code{Response} (HTML, JSON, redirection)
\end{enumerate}

\section{\texttt{CartController} - Gestion du Panier}

Le \controller{CartController} gere toutes les operations du panier d'achat.

\subsection{Structure et Attributs de Classe}

\begin{lstlisting}[style=phpstyle, caption={CartController.php - Declaration}]
<?php

namespace App\Controller;

use App\Entity\Product;
use App\Entity\Order;
use App\Entity\OrderItem;
use App\Repository\ProductRepository;
use App\Service\CartService;
use Doctrine\ORM\EntityManagerInterface;
use Symfony\Bundle\FrameworkBundle\Controller\AbstractController;
use Symfony\Component\HttpFoundation\Request;
use Symfony\Component\HttpFoundation\Response;
use Symfony\Component\Routing\Attribute\Route;
use Symfony\Component\Security\Http\Attribute\IsGranted;

#[Route('/cart')]
class CartController extends AbstractController
{
\end{lstlisting}

\textbf{Explications :}
\begin{itemize}
    \item \code{\#[Route('/cart')]} : Prefixe de route applique a toutes les actions de ce controleur.
    \item \code{AbstractController} : Classe parente fournissant des helpers (\code{render()}, \code{redirectToRoute()}, \code{json()}).
\end{itemize}

\subsection{Action \texttt{add()} - Ajout au Panier}

\begin{lstlisting}[style=phpstyle, caption={CartController::add()}]
#[Route('/add/{id}', name: 'app_cart_add', requirements: ['id' => '\d+'])]
#[IsGranted('ROLE_USER')]
public function add(int $id, ProductRepository $productRepository, 
                    CartService $cartService): Response
{
    $product = $productRepository->find($id);
    
    if (!$product) {
        $this->addFlash('error', 'Produit introuvable.');
        return $this->redirectToRoute('app_product_index');
    }
    
    if ($product->getStock() <= 0) {
        $this->addFlash('error', 'Ce produit est en rupture de stock.');
        return $this->redirectToRoute('app_product_show', ['id' => $id]);
    }
    
    // Ajouter le produit au panier (stockage optimise: ID seulement)
    $cartService->addProduct($id);
    
    $this->addFlash('success', 'Produit ajoute au panier avec succes !');
    
    return $this->redirectToRoute('app_product_show', ['id' => $id]);
}
\end{lstlisting}

\begin{table}[H]
\centering
\caption{Analyse de l'action \texttt{add()}}
\begin{tabularx}{\textwidth}{|l|X|}
\hline
\textbf{Element} & \textbf{Explication} \\
\hline
\hline
\code{requirements: ['id' => '\textbackslash d+']} & Validation regex : l'ID doit etre numerique \\
\hline
\code{\#[IsGranted('ROLE\_USER')]} & Seuls les utilisateurs connectes accedent a cette route \\
\hline
\code{ProductRepository \$productRepository} & Injection automatique de dependance (autowiring) \\
\hline
\code{addFlash()} & Message flash stocke en session, affiche une seule fois \\
\hline
\code{redirectToRoute()} & Redirection HTTP 302 vers une route nommee \\
\hline
\end{tabularx}
\end{table}

\subsection{Action \texttt{placeOrder()} - Validation de Commande}

\begin{lstlisting}[style=phpstyle, caption={CartController::placeOrder()}]
#[Route('/place-order', name: 'app_cart_place_order', methods: ['POST'])]
#[IsGranted('ROLE_USER')]
public function placeOrder(Request $request, CartService $cartService, 
                           EntityManagerInterface $em): Response
{
    // Verification du token CSRF
    $token = $request->request->get('_token');
    if (!$this->isCsrfTokenValid('place-order', $token)) {
        $this->addFlash('error', 'Erreur de securite. Veuillez reessayer.');
        return $this->redirectToRoute('app_cart_checkout');
    }
    
    // Creer la commande
    $order = new Order();
    $order->setUser($this->getUser());
    $order->setStatut('en_attente');
    
    // Creer les OrderItems
    $total = 0;
    foreach ($cartService->getCart() as $productId => $item) {
        $product = $item['product'];
        
        $orderItem = new OrderItem();
        $orderItem->setProduct($product);
        $orderItem->setQuantite($item['quantity']);
        $orderItem->setPrixUnitaire($product->getPrix());
        
        $order->addOrderItem($orderItem);
        
        $total += $product->getPrix() * $item['quantity'];
    }
    
    $order->setMontantTotal((string)$total);
    
    // Persister (cascade sur OrderItems)
    $em->persist($order);
    $em->flush();
    
    // Vider le panier
    $cartService->clear();
    
    return $this->redirectToRoute('app_profile_orders');
}
\end{lstlisting}

\section{Diagramme de Sequence - Processus de Commande}

\begin{figure}[H]
\centering
\begin{tikzpicture}[
    actor/.style={rectangle, draw=black, minimum width=2cm, minimum height=0.8cm, fill=blue!10},
    lifeline/.style={dashed, gray},
    message/.style={-{Stealth}, thick},
    returnmsg/.style={-{Stealth}, dashed, thick}
]
    % Actors
    \node[actor] (user) at (0,0) {Utilisateur};
    \node[actor] (ctrl) at (4,0) {CartController};
    \node[actor] (srv) at (8,0) {CartService};
    \node[actor] (db) at (12,0) {Database};
    
    % Lifelines
    \draw[lifeline] (user.south) -- ++(0,-10);
    \draw[lifeline] (ctrl.south) -- ++(0,-10);
    \draw[lifeline] (srv.south) -- ++(0,-10);
    \draw[lifeline] (db.south) -- ++(0,-10);
    
    % Messages
    \draw[message] (0,-1) -- node[above, font=\scriptsize] {1. POST /place-order} (4,-1);
    \draw[message] (4,-2) -- node[above, font=\scriptsize] {2. isEmpty()} (8,-2);
    \draw[returnmsg] (8,-2.5) -- node[above, font=\scriptsize] {false} (4,-2.5);
    \draw[message] (4,-3.5) -- node[above, font=\scriptsize] {3. getCart()} (8,-3.5);
    \draw[returnmsg] (8,-4) -- node[above, font=\scriptsize] {array} (4,-4);
    \draw[message] (4,-5) -- node[above, font=\scriptsize] {4. persist(Order)} (12,-5);
    \draw[message] (4,-6) -- node[above, font=\scriptsize] {5. flush()} (12,-6);
    \draw[returnmsg] (12,-6.5) -- node[above, font=\scriptsize] {OK} (4,-6.5);
    \draw[message] (4,-7.5) -- node[above, font=\scriptsize] {6. clear()} (8,-7.5);
    \draw[returnmsg] (4,-9) -- node[above, font=\scriptsize] {7. Redirect /profile/orders} (0,-9);
    
\end{tikzpicture}
\caption{Diagramme de Sequence - Validation de Commande}
\end{figure}

% ============================================================================
%                    CHAPITRE 5 : SERVICES
% ============================================================================
\chapter{Services - La Logique Metier}

\section{Architecture des Services}

Les \textbf{Services} dans Symfony sont des classes PHP reutilisables contenant la logique metier. Ils sont automatiquement injectes grace a l'\textbf{autowiring}.

\section{\texttt{CartService} - Gestion du Panier en Cookie}

Le \service{CartService} gere le panier cote client via des cookies encodes en Base64.

\begin{lstlisting}[style=phpstyle, caption={CartService.php - Proprietes}]
<?php

namespace App\Service;

use App\Repository\ProductRepository;
use Symfony\Component\HttpFoundation\RequestStack;
use Symfony\Component\HttpFoundation\Cookie;

class CartService
{
    private const CART_COOKIE_NAME = 'freshmarket_cart_token';
    private const COOKIE_LIFETIME = 60 * 60 * 24 * 30; // 30 jours
    
    private array $cart = [];

    public function __construct(
        private RequestStack $requestStack,
        private ProductRepository $productRepository
    ) {
        $this->loadCartFromCookie();
    }
\end{lstlisting}

\subsection{Methodes Detaillees}

\begin{table}[H]
\centering
\caption{Methodes du CartService}
\begin{tabularx}{\textwidth}{|l|l|X|}
\hline
\textbf{Methode} & \textbf{Retour} & \textbf{Description} \\
\hline
\hline
\code{addProduct(\$id, \$qty)} & bool & Ajoute un produit au panier \\
\hline
\code{removeProduct(\$id)} & void & Supprime un produit du panier \\
\hline
\code{updateQuantity(\$id, \$qty)} & void & Met a jour la quantite \\
\hline
\code{getCart()} & array & Retourne le panier avec objets Product \\
\hline
\code{getRawCart()} & array & Retourne le panier brut (IDs) \\
\hline
\code{getTotal()} & float & Calcule le montant total \\
\hline
\code{getCount()} & int & Nombre de produits differents \\
\hline
\code{isEmpty()} & bool & Verifie si le panier est vide \\
\hline
\code{clear()} & void & Vide le panier \\
\hline
\code{createCookie()} & Cookie & Genere le cookie a envoyer \\
\hline
\end{tabularx}
\end{table}

\begin{lstlisting}[style=phpstyle, caption={CartService.php - Methode getCart()}]
public function getCart(): array
{
    $cartWithProducts = [];
    foreach ($this->cart as $productId => $quantity) {
        $product = $this->productRepository->find($productId);
        
        if ($product && $product->getStock() > 0) {
            $cartWithProducts[$productId] = [
                'product' => $product,
                'quantity' => $quantity
            ];
        } else {
            // Produit inexistant ou rupture : on le retire
            $this->removeProduct($productId);
        }
    }
    
    return $cartWithProducts;
}
\end{lstlisting}

\textbf{Avantages de cette architecture :}
\begin{itemize}
    \item Le panier persiste 30 jours meme sans connexion.
    \item Les donnees en cookie sont legeres (uniquement IDs).
    \item La reconstruction est faite a la demande (lazy loading).
\end{itemize}

% ============================================================================
%                    CHAPITRE 6 : SECURITE
% ============================================================================
\chapter{Securite - Protection de l'Application}

\section{Configuration du Pare-feu Symfony}

Le fichier \texttt{config/packages/security.yaml} definit toute la strategie de securite :

\begin{lstlisting}[style=yamlstyle, caption={security.yaml}]
security:
    password_hashers:
        Symfony\Component\Security\Core\User\PasswordAuthenticatedUserInterface: 'auto'
        App\Entity\User:
            algorithm: auto  # bcrypt ou argon2 selon le serveur
    
    providers:
        app_user_provider:
            entity:
                class: App\Entity\User
                property: email  # Identifiant de connexion
    
    firewalls:
        main:
            lazy: true
            provider: app_user_provider
            custom_authenticators:
                - App\Security\GoogleAuthenticator
                - App\Security\FacebookAuthenticator
            form_login:
                login_path: app_login
                check_path: app_login
                enable_csrf: true
                username_parameter: email
                password_parameter: password
            logout:
                path: app_logout
                target: app_home
            remember_me:
                secret: '%kernel.secret%'
                lifetime: 604800  # 7 jours
    
    access_control:
        - { path: ^/admin, roles: ROLE_ADMIN }
        - { path: ^/profile, roles: ROLE_USER }
        - { path: ^/cart, roles: ROLE_USER }
\end{lstlisting}

\section{OAuth 2.0 - Authentification Sociale}

\subsection{GoogleAuthenticator - Flux Complet}

\begin{lstlisting}[style=phpstyle, caption={GoogleAuthenticator.php - authenticate()}]
public function authenticate(Request $request): Passport
{
    $client = $this->clientRegistry->getClient('google');
    $accessToken = $this->fetchAccessToken($client);
    
    /** @var GoogleUser $googleUser */
    $googleUser = $client->fetchUserFromToken($accessToken);

    return new SelfValidatingPassport(
        new UserBadge($googleUser->getEmail(), function () use ($googleUser) {
            return $this->getOrCreateUser($googleUser);
        }),
        [new RememberMeBadge()]
    );
}
\end{lstlisting}

\begin{lstlisting}[style=phpstyle, caption={GoogleAuthenticator.php - getOrCreateUser()}]
private function getOrCreateUser(GoogleUser $googleUser): User
{
    $googleId = $googleUser->getId();
    $email = $googleUser->getEmail();

    // 1. Chercher par Google ID
    $existingUser = $this->userRepository->findOneBy(['googleId' => $googleId]);
    if ($existingUser instanceof User) {
        return $existingUser;
    }

    // 2. Chercher par email (lier compte existant)
    $userByEmail = $this->userRepository->findOneBy(['email' => $email]);
    if ($userByEmail instanceof User) {
        $userByEmail->setGoogleId($googleId);
        $this->entityManager->flush();
        return $userByEmail;
    }

    // 3. Creer un nouvel utilisateur
    $user = new User();
    $user->setEmail($email);
    $user->setGoogleId($googleId);
    $user->setPrenom($googleUser->getFirstName() ?: 'Utilisateur');
    $user->setNom($googleUser->getLastName() ?: 'Google');
    
    // Mot de passe aleatoire (non utilisable directement)
    $randomPassword = bin2hex(random_bytes(16));
    $user->setPassword($this->passwordHasher->hashPassword($user, $randomPassword));

    $this->entityManager->persist($user);
    $this->entityManager->flush();

    return $user;
}
\end{lstlisting}

\section{Headers de Securite HTTP}

Le \code{SecurityHeadersSubscriber} injecte des headers de protection :

\begin{lstlisting}[style=phpstyle, caption={SecurityHeadersSubscriber.php}]
public function onKernelResponse(ResponseEvent $event): void
{
    $response = $event->getResponse();
    $headers = $response->headers;

    // Protection XSS
    $headers->set('X-XSS-Protection', '1; mode=block');

    // Protection Clickjacking
    $headers->set('X-Frame-Options', 'DENY');

    // Protection MIME Sniffing
    $headers->set('X-Content-Type-Options', 'nosniff');

    // Referrer Policy
    $headers->set('Referrer-Policy', 'strict-origin-when-cross-origin');

    // Content Security Policy
    $csp = implode('; ', [
        "default-src 'self'",
        "script-src 'self' 'unsafe-inline' https://cdn.jsdelivr.net",
        "style-src 'self' 'unsafe-inline' https://cdn.jsdelivr.net",
        "img-src 'self' data: https:",
        "frame-ancestors 'none'",
    ]);
    $headers->set('Content-Security-Policy', $csp);
}
\end{lstlisting}

\begin{table}[H]
\centering
\caption{Headers de Securite et leurs Protections}
\begin{tabularx}{\textwidth}{|l|X|}
\hline
\textbf{Header} & \textbf{Protection Fournie} \\
\hline
\hline
\code{X-XSS-Protection} & Active le filtre XSS du navigateur \\
\hline
\code{X-Frame-Options: DENY} & Empeche l'inclusion dans une iframe (Clickjacking) \\
\hline
\code{X-Content-Type-Options} & Empeche le MIME sniffing \\
\hline
\code{Content-Security-Policy} & Controle les sources de scripts/styles autorisees \\
\hline
\code{Referrer-Policy} & Limite les informations envoyees a des sites tiers \\
\hline
\code{Strict-Transport-Security} & Force HTTPS (production uniquement) \\
\hline
\end{tabularx}
\end{table}

\section{Protection CSRF}

Symfony active la protection CSRF par defaut sur les formulaires :

\begin{lstlisting}[style=twigstyle, caption={Template Twig avec CSRF}]
<form method="post" action="{{ path('app_cart_place_order') }}">
    <input type="hidden" name="_token" 
           value="{{ csrf_token('place-order') }}">
    <!-- ... champs du formulaire ... -->
    <button type="submit">Valider la commande</button>
</form>
\end{lstlisting}

\begin{lstlisting}[style=phpstyle, caption={Validation CSRF cote controleur}]
$token = $request->request->get('_token');
if (!$this->isCsrfTokenValid('place-order', $token)) {
    throw new AccessDeniedException('Token CSRF invalide');
}
\end{lstlisting}

% ============================================================================
%                    CHAPITRE 7 : REPOSITORIES
% ============================================================================
\chapter{Repositories - Les Requetes Personnalisees}

\section{Role des Repositories}

Les \textbf{Repositories} encapsulent les requetes SQL complexes. Ils heritent de \code{ServiceEntityRepository} fourni par Doctrine.

\section{\texttt{ProductRepository} - Exemple Complet}

\begin{lstlisting}[style=phpstyle, caption={ProductRepository.php}]
<?php

namespace App\Repository;

use App\Entity\Product;
use Doctrine\Bundle\DoctrineBundle\Repository\ServiceEntityRepository;
use Doctrine\Persistence\ManagerRegistry;

class ProductRepository extends ServiceEntityRepository
{
    public function __construct(ManagerRegistry $registry)
    {
        parent::__construct($registry, Product::class);
    }

    /**
     * Trouve tous les produits disponibles et en stock
     */
    public function findAvailable(): array
    {
        return $this->createQueryBuilder('p')
            ->andWhere('p.disponible = :disponible')
            ->andWhere('p.stock > 0')
            ->setParameter('disponible', true)
            ->orderBy('p.nom', 'ASC')
            ->getQuery()
            ->getResult();
    }

    /**
     * Recherche par categorie
     */
    public function findByCategory(string $category): array
    {
        return $this->createQueryBuilder('p')
            ->andWhere('p.categorie = :category')
            ->andWhere('p.disponible = :disponible')
            ->setParameter('category', $category)
            ->setParameter('disponible', true)
            ->orderBy('p.nom', 'ASC')
            ->getQuery()
            ->getResult();
    }

    /**
     * Recherche textuelle dans nom et description
     */
    public function searchProducts(string $query): array
    {
        return $this->createQueryBuilder('p')
            ->andWhere('p.nom LIKE :query OR p.description LIKE :query')
            ->andWhere('p.disponible = :disponible')
            ->setParameter('query', '%'.$query.'%')
            ->setParameter('disponible', true)
            ->orderBy('p.nom', 'ASC')
            ->getQuery()
            ->getResult();
    }
}
\end{lstlisting}

\textbf{Avantages du QueryBuilder :}
\begin{itemize}
    \item Requetes securisees (preparation automatique des parametres).
    \item Code lisible et maintenable.
    \item Generation SQL optimisee par Doctrine.
\end{itemize}

% ============================================================================
%                    CHAPITRE 8 : FORMULAIRES
% ============================================================================
\chapter{Formulaires Symfony}

\section{Architecture des Formulaires}

Symfony utilise le pattern \textbf{Form Type} pour definir des formulaires reutilisables.

\section{\texttt{RegistrationFormType} - Inscription}

\begin{lstlisting}[style=phpstyle, caption={RegistrationFormType.php}]
<?php

namespace App\Form;

use App\Entity\User;
use Symfony\Component\Form\AbstractType;
use Symfony\Component\Form\Extension\Core\Type\CheckboxType;
use Symfony\Component\Form\Extension\Core\Type\EmailType;
use Symfony\Component\Form\Extension\Core\Type\PasswordType;
use Symfony\Component\Form\Extension\Core\Type\RepeatedType;
use Symfony\Component\Form\FormBuilderInterface;
use Symfony\Component\OptionsResolver\OptionsResolver;
use Symfony\Component\Validator\Constraints\IsTrue;
use Symfony\Component\Validator\Constraints\Length;
use Symfony\Component\Validator\Constraints\NotBlank;

class RegistrationFormType extends AbstractType
{
    public function buildForm(FormBuilderInterface $builder, array $options): void
    {
        $builder
            ->add('email', EmailType::class, [
                'label' => 'Email',
                'attr' => [
                    'class' => 'form-control',
                    'placeholder' => 'votre@email.com'
                ]
            ])
            ->add('nom', TextType::class, [
                'label' => 'Nom',
                'attr' => ['class' => 'form-control']
            ])
            ->add('prenom', TextType::class, [
                'label' => 'Prenom',
                'attr' => ['class' => 'form-control']
            ])
            ->add('agreeTerms', CheckboxType::class, [
                'label' => 'J\'accepte les conditions',
                'mapped' => false,  // Non lie a l'entite
                'constraints' => [
                    new IsTrue([
                        'message' => 'Vous devez accepter les conditions.',
                    ]),
                ],
            ])
            ->add('plainPassword', RepeatedType::class, [
                'type' => PasswordType::class,
                'mapped' => false,
                'first_options' => [
                    'label' => 'Mot de passe',
                    'constraints' => [
                        new NotBlank(['message' => 'Mot de passe requis']),
                        new Length([
                            'min' => 6,
                            'minMessage' => 'Minimum 6 caracteres',
                        ]),
                    ],
                ],
                'second_options' => [
                    'label' => 'Confirmer le mot de passe',
                ],
                'invalid_message' => 'Les mots de passe doivent correspondre.',
            ]);
    }

    public function configureOptions(OptionsResolver $resolver): void
    {
        $resolver->setDefaults([
            'data_class' => User::class,
        ]);
    }
}
\end{lstlisting}

% ============================================================================
%                    CHAPITRE 9 : TEMPLATES TWIG
% ============================================================================
\chapter{Templates Twig - La Vue}

\section{Heritage de Templates}

Twig utilise un systeme d'\textbf{heritage} avec \code{extends} et \code{block}.

\begin{lstlisting}[style=twigstyle, caption={base.html.twig - Structure}]
<!DOCTYPE html>
<html lang="fr">
<head>
    <meta charset="UTF-8">
    <meta name="viewport" content="width=device-width, initial-scale=1.0">
    <title>Bienvenue - FreshMarket</title>
    
    
        <link href="https://cdn.jsdelivr.net/..." rel="stylesheet">
        <link href="{{ asset('styles/app.css') }}" rel="stylesheet">
    
</head>
<body>
    
    
    <main>
        
    </main>
    
    
    
    
        <script src="https://cdn.jsdelivr.net/..."></script>
    
</body>
</html>
\end{lstlisting}

\section{Variables et Filtres}

\begin{lstlisting}[style=twigstyle, caption={Affichage de donnees}]
{# Variables #}
{{ product.nom }}
{{ product.prix|number_format(2, ',', ' ') }} EUR

{# Conditions #}

    <span class="badge bg-success">En stock</span>

    <span class="badge bg-danger">Rupture</span>


{# Boucles #}

    <tr>
        <td>{{ item.product.nom }}</td>
        <td>{{ item.quantity }}</td>
        <td>{{ item.product.prix * item.quantity }} EUR</td>
    </tr>

    <tr><td colspan="3">Panier vide</td></tr>

\end{lstlisting}

% ============================================================================
%                    CHAPITRE 10 : DEPLOIEMENT
% ============================================================================
\chapter{Deploiement et Configuration}

\section{Fichiers de Configuration}

\begin{table}[H]
\centering
\caption{Fichiers d'Environnement}
\begin{tabularx}{\textwidth}{|l|X|}
\hline
\textbf{Fichier} & \textbf{Usage} \\
\hline
\hline
\code{.env} & Variables par defaut (versionnees) \\
\hline
\code{.env.local} & Variables locales (non versionnees) \\
\hline
\code{.env.prod} & Variables de production \\
\hline
\code{.env.test} & Variables pour les tests \\
\hline
\end{tabularx}
\end{table}

\section{Docker}

\begin{lstlisting}[style=yamlstyle, caption={docker-compose.yml}]
version: '3.8'
services:
  web:
    build: .
    ports:
      - "8080:80"
    volumes:
      - .:/var/www/html
    depends_on:
      - database

  database:
    image: mariadb:10.11
    environment:
      MYSQL_ROOT_PASSWORD: secret
      MYSQL_DATABASE: freshmarket
    volumes:
      - db_data:/var/lib/mysql

volumes:
  db_data:
\end{lstlisting}

% ============================================================================
%                           CONCLUSION
% ============================================================================
\chapter{Conclusion}

Ce rapport technique a presente en detail l'architecture et l'implementation du projet \textbf{FreshMarket}, une application e-commerce complete developpee avec Symfony 7.3. 

\section{Recapitulatif des Points Cles}

\begin{itemize}
    \item \textbf{Architecture MVC} : Separation stricte entre Entites, Controleurs et Templates.
    \item \textbf{Doctrine ORM} : Mapping objet-relationnel avec validation integree.
    \item \textbf{Securite Multicouche} : OAuth, CSRF, Headers HTTP, Hashage bcrypt/argon2.
    \item \textbf{Services Reutilisables} : CartService encapsulant la logique metier.
    \item \textbf{Frontend Moderne} : Stimulus/Turbo pour une experience SPA legere.
\end{itemize}

\section{Points d'Amelioration Suggeres}

\begin{enumerate}
    \item Implementer la decrementation automatique du stock lors d'une commande.
    \item Ajouter un systeme de paiement (Stripe, PayPal).
    \item Mettre en place des tests unitaires et fonctionnels avec PHPUnit.
    \item Configurer un pipeline CI/CD (GitHub Actions, GitLab CI).
\end{enumerate}

\appendix
\chapter{Glossaire}

\begin{description}
    \item[ORM] Object-Relational Mapping - Technique de correspondance entre objets et tables.
    \item[CSRF] Cross-Site Request Forgery - Attaque forcant un utilisateur a executer des actions.
    \item[XSS] Cross-Site Scripting - Injection de scripts malveillants.
    \item[OAuth] Protocole d'autorisation permettant l'authentification via des tiers.
    \item[Autowiring] Injection automatique de dependances par Symfony.
\end{description}

\end{document}
