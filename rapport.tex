\documentclass[a4paper,12pt,twoside]{report}
\usepackage[utf8]{inputenc}
\usepackage[T1]{fontenc}
\usepackage[french]{babel}
\usepackage{geometry}
\usepackage{graphicx}
\usepackage{hyperref}
\usepackage{listings}
\usepackage{xcolor}
\usepackage{tikz}
\usepackage{float}
\usepackage{fancyhdr}
\usepackage{tocloft}
\usepackage{tabularx}
\usepackage{booktabs}
\usepackage{enumitem}
\usepackage{amsmath}
\usepackage{amssymb}
\usepackage{fontawesome5}

% Configuration des marges
\geometry{
    top=2.5cm,
    bottom=2.5cm,
    left=2.5cm,
    right=2.5cm,
    headheight=15pt
}

% Configuration des headers/footers
\pagestyle{fancy}
\fancyhf{}
\fancyhead[LE,RO]{\thepage}
\fancyhead[LO]{\nouppercase{\leftmark}}
\fancyhead[RE]{\nouppercase{Rapport Technique - Synf Project}}
\renewcommand{\headrulewidth}{0.4pt}
\renewcommand{\footrulewidth}{0pt}

% Configuration des liens hypertextes
\hypersetup{
    colorlinks=true,
    linkcolor=blue!70!black,
    filecolor=magenta,      
    urlcolor=blue!60!black,
    citecolor=green!50!black,
    pdftitle={Rapport Technique Complet - Synf Project E-commerce Symfony 7},
    pdfauthor={Équipe de Développement},
    pdfsubject={Documentation Technique Complète},
    pdfkeywords={Symfony, PHP, E-commerce, Doctrine, OAuth, Sécurité}
}

% Couleurs personnalisées
\definecolor{codegreen}{rgb}{0,0.6,0}
\definecolor{codegray}{rgb}{0.5,0.5,0.5}
\definecolor{codepurple}{rgb}{0.58,0,0.82}
\definecolor{backcolour}{rgb}{0.97,0.97,0.95}
\definecolor{symfonycolor}{RGB}{0, 0, 0}
\definecolor{phpcolor}{RGB}{119, 123, 180}
\definecolor{doctrinecolor}{RGB}{252, 108, 38}
\definecolor{twigcolor}{RGB}{186, 205, 34}

% Configuration pour le code source PHP
\lstdefinestyle{phpstyle}{
    language=PHP,
    backgroundcolor=\color{backcolour},   
    commentstyle=\color{codegreen}\itshape,
    keywordstyle=\color{phpcolor}\bfseries,
    numberstyle=\tiny\color{codegray},
    stringstyle=\color{codepurple},
    basicstyle=\ttfamily\footnotesize,
    breakatwhitespace=false,         
    breaklines=true,                 
    captionpos=b,                    
    keepspaces=true,                 
    numbers=left,                    
    numbersep=5pt,                  
    showspaces=false,                
    showstringspaces=false,
    showtabs=false,                  
    tabsize=2,
    frame=single,
    rulecolor=\color{codegray},
    morekeywords={class, function, public, private, protected, static, return, new, use, namespace, extends, implements, readonly, ?int, ?string, ?bool, array, Collection, self, mixed},
    emph={Entity, Repository, Controller, Service, Form},
    emphstyle=\color{doctrinecolor}\bfseries
}

% Configuration pour YAML
\lstdefinestyle{yamlstyle}{
    backgroundcolor=\color{backcolour},
    basicstyle=\ttfamily\footnotesize,
    breaklines=true,
    numbers=left,
    numbersep=5pt,
    frame=single,
    rulecolor=\color{codegray},
    keywordstyle=\color{blue!70!black}\bfseries,
    commentstyle=\color{codegreen}\itshape,
    stringstyle=\color{codepurple}
}

% Configuration pour Twig
\lstdefinestyle{twigstyle}{
    backgroundcolor=\color{backcolour},
    basicstyle=\ttfamily\footnotesize,
    breaklines=true,
    numbers=left,
    numbersep=5pt,
    frame=single,
    rulecolor=\color{codegray},
    keywordstyle=\color{twigcolor}\bfseries,
    commentstyle=\color{codegreen}\itshape,
    stringstyle=\color{codepurple},
    morekeywords={block, endblock, extends, if, endif, for, endfor, include}
}

\lstset{style=phpstyle}

% TikZ libraries
\usetikzlibrary{shapes,arrows.meta,positioning,fit,backgrounds,shadows.blur,calc,decorations.pathreplacing}

% Custom commands
\newcommand{\code}[1]{\texttt{\textcolor{codepurple}{#1}}}
\newcommand{\entity}[1]{\texttt{\textcolor{doctrinecolor}{\bfseries #1}}}
\newcommand{\controller}[1]{\texttt{\textcolor{blue!70!black}{\bfseries #1}}}
\newcommand{\service}[1]{\texttt{\textcolor{codegreen}{\bfseries #1}}}

\begin{document}

% ============================================================================
%                           PAGE DE GARDE
% ============================================================================
\begin{titlepage}
    \begin{center}
        \vspace*{1cm}
        
        \begin{tikzpicture}
            \node[draw=blue!70!black, line width=3pt, rounded corners=15pt, inner sep=20pt, fill=blue!10, drop shadow] {
                \begin{minipage}{0.8\textwidth}
                    \centering
                    \Huge\bfseries\textcolor{blue!70!black}{Rapport Technique Détaillé}
                    
                    \vspace{0.5cm}
                    \Large\textcolor{gray!70}{Projet FreshMarket E-commerce}
                \end{minipage}
            };
        \end{tikzpicture}
        
        \vspace{1.5cm}
        
        \begin{tikzpicture}
            % Logo stylisé
            \node[circle, draw=blue!70!black, line width=3pt, minimum size=3.5cm, fill=blue!15, drop shadow] (logo) {};
            \node at (logo) {\Huge\textcolor{blue!60!black}{\faShoppingCart}};
        \end{tikzpicture}
        
        \vspace{1.5cm}
        
        \Large
        \begin{tabular}{rl}
            \textbf{Framework :} & Symfony 7.3 \\
            \textbf{Langage :} & PHP 8.x \\
            \textbf{Base de Données :} & MySQL / MariaDB \\
            \textbf{ORM :} & Doctrine 3.x \\
            \textbf{Template :} & Twig 3.x \\
        \end{tabular}
        
        \vfill
        
        \begin{tikzpicture}
            \node[draw=gray!70, line width=2pt, rounded corners=8pt, inner sep=15pt, fill=gray!10, drop shadow] {
                \begin{minipage}{0.6\textwidth}
                    \centering
                    \large
                    \textbf{Auteur :} Équipe de Développement\\[0.3cm]
                    \textbf{Version :} 1.0\\[0.3cm]
                    \textbf{Date :} \today
                \end{minipage}
            };
        \end{tikzpicture}
        
    \end{center}
\end{titlepage}

% ============================================================================
%                           TABLE DES MATIÈRES
% ============================================================================
\tableofcontents
\newpage

% ============================================================================
%                           CHAPITRE 1 : INTRODUCTION
% ============================================================================
\chapter{Introduction Générale}

\section{Présentation du Projet}

Le projet \textbf{FreshMarket} (aussi connu sous le nom de code \texttt{synf\_project}) est une application web complète de type \textbf{e-commerce} et \textbf{système de réservation} développée avec le framework \textbf{Symfony 7.3}. Cette plateforme permet aux utilisateurs de :

\begin{itemize}[leftmargin=2cm]
    \item[\faUserPlus] S'inscrire et gérer leur profil utilisateur
    \item[\faGoogle] Se connecter via OAuth (Google, Facebook)
    \item[\faShoppingBasket] Consulter un catalogue de produits avec gestion du stock
    \item[\faShoppingCart] Ajouter des produits au panier et passer des commandes
    \item[\faCalendarCheck] Effectuer des réservations de services
    \item[\faChartBar] Consulter des statistiques personnalisées
\end{itemize}

\section{Objectif de ce Rapport}

Ce document technique a pour objectif de fournir une \textbf{documentation exhaustive} permettant à tout développeur de comprendre \textbf{100\% du projet techniquement}. Chaque fichier, chaque fonction, chaque relation est expliquée en détail avec des exemples de code commentés.

\section{Technologies Utilisées - Vue d'Ensemble}

\begin{table}[H]
\centering
\caption{Stack Technologique Complète}
\begin{tabularx}{\textwidth}{|l|X|l|}
\hline
\textbf{Catégorie} & \textbf{Technologie} & \textbf{Version} \\
\hline
\hline
Framework & Symfony & 7.3 \\
\hline
Langage & PHP & 8.x \\
\hline
ORM & Doctrine (DBAL + ORM) & 3.x \\
\hline
Template Engine & Twig & 3.x \\
\hline
Authentification & Symfony Security + OAuth2 & N/A \\
\hline
OAuth Library & knpuniversity/oauth2-client-bundle & 2.19 \\
\hline
Frontend JS & Stimulus (Symfony UX) & 2.30 \\
\hline
SPA Enhancement & Symfony UX Turbo & 2.30 \\
\hline
CSS Framework & Bootstrap 5 & 5.3 \\
\hline
Validation & Symfony Validator & 7.3 \\
\hline
Mailing & Symfony Mailer & 7.4 \\
\hline
Asset Management & Symfony AssetMapper & 7.3 \\
\hline
\end{tabularx}
\end{table}

% ============================================================================
%                    CHAPITRE 2 : MÉTHODOLOGIE AGILE SCRUM
% ============================================================================
\chapter{Méthodologie Agile Scrum}

\section{Présentation de la Méthodologie}

Le développement du projet \textbf{FreshMarket} a été réalisé en suivant la méthodologie \textbf{Agile Scrum}. Cette approche itérative et incrémentale permet une adaptation continue aux besoins du projet et une livraison régulière de fonctionnalités opérationnelles.

\subsection{Les Principes Scrum Appliqués}

\begin{figure}[H]
\centering
\begin{tikzpicture}[
    node distance=2cm,
    box/.style={rectangle, draw=blue!70!black, line width=2pt, rounded corners=8pt, minimum width=3.5cm, minimum height=1.5cm, align=center, fill=blue!15, drop shadow, font=\bfseries\small},
    arrow/.style={-{Stealth[length=3mm]}, line width=1.5pt, blue!60!black},
    cycle/.style={-{Stealth[length=3mm]}, line width=1.5pt, orange!70!black, bend right=30}
]
    % Product Backlog
    \node[box, fill=green!25] (backlog) at (0,0) {Product\\Backlog};
    
    % Sprint Planning
    \node[box, fill=yellow!30] (planning) at (4,0) {Sprint\\Planning};
    
    % Sprint Backlog
    \node[box, fill=orange!25] (sprint) at (8,0) {Sprint\\Backlog};
    
    % Daily Scrum
    \node[box, fill=red!25] (daily) at (12,0) {Daily\\Scrum};
    
    % Sprint (2 semaines)
    \node[box, fill=purple!25, minimum width=4.5cm] (iteration) at (10,-3.5) {Sprint\\(2 jours)};
    
    % Increment
    \node[box, fill=blue!35] (increment) at (4,-3.5) {Incrément\\Livrable};
    
    % Review & Retro
    \node[box, fill=cyan!25, minimum width=4cm] (review) at (0,-3.5) {Sprint Review\\+ Rétrospective};
    
    % Arrows
    \draw[arrow] (backlog) -- (planning);
    \draw[arrow] (planning) -- (sprint);
    \draw[arrow] (sprint) -- (daily);
    \draw[arrow] (daily) to[out=-90, in=0] (iteration);
    \draw[arrow] (iteration) -- (increment);
    \draw[arrow] (increment) -- (review);
    \draw[cycle] (review) to[bend right=35] (backlog);
    
\end{tikzpicture}
\caption{Cycle Scrum Appliqué au Projet FreshMarket}
\end{figure}

\begin{description}
    \item[\faUsers\ Équipe Scrum :] Développeur(s) full-stack, Scrum Master, Product Owner
    \item[\faCalendarAlt\ Sprints :] Itérations de 2 jours avec objectifs définis
    \item[\faClipboardList\ Product Backlog :] Liste priorisée des User Stories
    \item[\faChartLine\ Vélocité :] Estimation en Story Points pour mesurer la productivité
\end{description}

\section{Planification des Sprints}

Le projet a été organisé en \textbf{2 sprints} successifs, permettant une livraison progressive des fonctionnalités :

\begin{table}[H]
\centering
\caption{Organisation des Sprints}
\begin{tabularx}{\textwidth}{|l|c|c|X|c|}
\hline
\textbf{Sprint} & \textbf{Début} & \textbf{Fin} & \textbf{Objectif} & \textbf{Vélocité} \\
\hline
\hline
Sprint 1 & 23/12/2025 & 24/12/2025 & MVP : liste produits, panier, authentification de base, sécurité & 27 pts \\
\hline
Sprint 2 & 25/12/2025 & 26/12/2025 & Checkout, admin commandes, CI/tests, dashboard, OAuth & 28 pts \\
\hline
\end{tabularx}
\end{table}

\subsection{Sprint 1 : MVP (Minimum Viable Product)}

\begin{figure}[H]
\centering
\begin{tikzpicture}[
    story/.style={rectangle, draw=green!70!black, line width=2pt, rounded corners=8pt, minimum width=4cm, minimum height=1cm, fill=green!20, font=\small\bfseries, drop shadow},
    done/.style={rectangle, draw=blue!70!black, line width=2.5pt, rounded corners=8pt, minimum width=4cm, minimum height=1cm, fill=blue!25, font=\small\bfseries, drop shadow}
]
    % Timeline
    \draw[thick, gray] (0,0) -- (12,0);
    \foreach \x in {0,3,6,9,12} {
        \draw[thick, gray] (\x,-0.1) -- (\x,0.1);
    }
    \node[below] at (0,-0.2) {\scriptsize 23/12};
    \node[below] at (6,-0.2) {\scriptsize 23/12};
    \node[below] at (12,-0.2) {\scriptsize 24/12};
    
    % Stories
    \node[done] at (2,1) {US-01 : Liste Produits (3 pts)};
    \node[done] at (2,2) {US-02 : Détail Produit (2 pts)};
    \node[done] at (6,1) {US-04 : Gestion Panier (5 pts)};
    \node[done] at (6,2) {US-07 : Auth \& Rôles (5 pts)};
    \node[done] at (10,1) {US-12 : Sécurité (3 pts)};
    \node[done] at (10,2) {US-13 : Déploiement (3 pts)};
    \node[done] at (6,3) {US-09 : Profil (3 pts)};
    \node[done] at (10,3) {US-03 : Recherche (3 pts)};
    
\end{tikzpicture}
\caption{Timeline du Sprint 1}
\end{figure}

\subsection{Sprint 2 : Fonctionnalités Avancées}

\begin{figure}[H]
\centering
\begin{tikzpicture}[
    story/.style={rectangle, draw=orange!70!black, line width=2pt, rounded corners=8pt, minimum width=4cm, minimum height=1cm, fill=orange!20, font=\small\bfseries, drop shadow},
    done/.style={rectangle, draw=purple!70!black, line width=2.5pt, rounded corners=8pt, minimum width=4cm, minimum height=1cm, fill=purple!25, font=\small\bfseries, drop shadow}
]
    % Timeline
    \draw[thick, gray] (0,0) -- (12,0);
    \foreach \x in {0,4,8,12} {
        \draw[thick, gray] (\x,-0.1) -- (\x,0.1);
    }
    \node[below] at (0,-0.2) {\scriptsize 25/12};
    \node[below] at (6,-0.2) {\scriptsize 25/12};
    \node[below] at (12,-0.2) {\scriptsize 26/12};
    
    % Stories
    \node[done] at (2,1) {US-05 : Checkout (8 pts)};
    \node[done] at (6,1) {US-06 : Admin Commandes (5 pts)};
    \node[done] at (10,1) {US-08 : OAuth (5 pts)};
    \node[done] at (6,2) {US-10 : Dashboard Stats (5 pts)};
    \node[done] at (10,2) {US-11 : Tests \& CI (5 pts)};
    
\end{tikzpicture}
\caption{Timeline du Sprint 2}
\end{figure}

\section{Product Backlog - User Stories}

Le Product Backlog contient l'ensemble des \textbf{User Stories} du projet, priorisées et estimées en Story Points :

\begin{table}[H]
\centering
\caption{Product Backlog Complet}
\resizebox{\textwidth}{!}{%
\begin{tabular}{|l|l|c|c|l|l|}
\hline
\textbf{ID} & \textbf{Titre} & \textbf{Points} & \textbf{Priorité} & \textbf{Sprint} & \textbf{Labels} \\
\hline
\hline
US-01 & Affichage liste produits & 3 & Haute & Sprint 1 & product, ui \\
\hline
US-02 & Page détail produit & 2 & Haute & Sprint 1 & product, ui \\
\hline
US-03 & Recherche et filtrage & 3 & Moyenne & Sprint 1 & product, search \\
\hline
US-04 & Gestion panier & 5 & Haute & Sprint 1 & cart \\
\hline
US-05 & Checkout / Création commande & 8 & Haute & Sprint 2 & checkout, order \\
\hline
US-06 & Gestion commandes (admin) & 5 & Haute & Sprint 2 & admin, orders \\
\hline
US-07 & Authentification \& rôles & 5 & Haute & Sprint 1 & security, auth \\
\hline
US-08 & Login OAuth (Google/Facebook) & 5 & Moyenne & Sprint 2 & auth, oauth \\
\hline
US-09 & Gestion profil utilisateur & 3 & Moyenne & Sprint 1 & profile, ui \\
\hline
US-10 & Dashboard statistiques (admin) & 5 & Moyenne & Sprint 2 & admin, stats \\
\hline
US-11 & Tests automatiques \& CI & 5 & Haute & Sprint 2 & ci, test \\
\hline
US-12 & Sécurité / hardening & 3 & Haute & Sprint 1 & security \\
\hline
US-13 & Déploiement \& documentation & 3 & Moyenne & Sprint 1 & devops, deploy \\
\hline
\multicolumn{2}{|r|}{\textbf{Total Story Points}} & \textbf{55} & & & \\
\hline
\end{tabular}
}
\end{table}

\section{Détail des User Stories Principales}

\subsection{US-04 : Gestion Panier (5 points)}

\begin{table}[H]
\centering
\begin{tabularx}{\textwidth}{|l|X|}
\hline
\textbf{Description} & En tant qu'utilisateur, je veux ajouter/supprimer/modifier la quantité des produits dans le panier. \\
\hline
\textbf{Critères d'acceptation} & 
\begin{minipage}[t]{\linewidth}
\vspace{2pt}
\begin{itemize}[leftmargin=*, nosep]
    \item Persistance du panier (session/cookie)
    \item Calcul automatique du total
    \item Notifications UI après chaque action
    \item Validation du stock disponible
\end{itemize}
\vspace{2pt}
\end{minipage} \\
\hline
\textbf{Implémentation} & \code{CartService.php}, \code{CartController.php} \\
\hline
\end{tabularx}
\end{table}

\subsection{US-05 : Checkout / Création Commande (8 points)}

\begin{table}[H]
\centering
\begin{tabularx}{\textwidth}{|l|X|}
\hline
\textbf{Description} & En tant qu'utilisateur, je veux finaliser ma commande et recevoir une confirmation par e-mail. \\
\hline
\textbf{Critères d'acceptation} & 
\begin{minipage}[t]{\linewidth}
\vspace{2pt}
\begin{itemize}[leftmargin=*, nosep]
    \item Formulaire d'adresse de livraison
    \item Création des entités Order + OrderItem
    \item Envoi d'email de confirmation
    \item Tests end-to-end fonctionnels
\end{itemize}
\vspace{2pt}
\end{minipage} \\
\hline
\textbf{Implémentation} & \code{Order.php}, \code{OrderItem.php}, \code{CartController::placeOrder()} \\
\hline
\end{tabularx}
\end{table}

\subsection{US-08 : Login OAuth Google/Facebook (5 points)}

\begin{table}[H]
\centering
\begin{tabularx}{\textwidth}{|l|X|}
\hline
\textbf{Description} & En tant qu'utilisateur, je veux me connecter via Google ou Facebook. \\
\hline
\textbf{Critères d'acceptation} & 
\begin{minipage}[t]{\linewidth}
\vspace{2pt}
\begin{itemize}[leftmargin=*, nosep]
    \item Boutons OAuth sur la page de connexion
    \item Mapping automatique d'un compte existant (par email)
    \item Création de compte si nouveau
    \item Tests du flux OAuth complet
\end{itemize}
\vspace{2pt}
\end{minipage} \\
\hline
\textbf{Implémentation} & \code{GoogleAuthenticator.php}, \code{FacebookAuthenticator.php} \\
\hline
\end{tabularx}
\end{table}

\section{Burndown Chart}

\begin{figure}[H]
\centering
\begin{tikzpicture}[scale=1.5]
    % Axes
    \draw[thick, ->, line width=1.5pt] (0,0) -- (5,0) node[right, font=\bfseries] {Jours};
    \draw[thick, ->, line width=1.5pt] (0,0) -- (0,7) node[above, font=\bfseries] {Story Points};
    
    % Grid
    \foreach \y in {1,...,6} {
        \draw[gray!30, thin] (0,\y) -- (4,\y);
        \node[left, font=\small\bfseries] at (-0.2,\y) {\pgfmathparse{int(\y*10)}\pgfmathresult};
    }
    \foreach \x in {0,...,4} {
        \draw[gray!20, thin] (\x,0) -- (\x,6);
    }
    
    % Sprint 1 (jours 1-2)
    \node[below, font=\small\bfseries, text=green!70!black] at (0,-0.3) {J1};
    \node[below, font=\small\bfseries, text=green!70!black] at (1,-0.6) {Sprint 1};
    \node[below, font=\small\bfseries, text=green!70!black] at (2,-0.3) {J2};
    
    % Sprint 2 (jours 3-4)
    \node[below, font=\small\bfseries, text=orange!70!black] at (2.5,-0.3) {J3};
    \node[below, font=\small\bfseries, text=orange!70!black] at (3,-0.6) {Sprint 2};
    \node[below, font=\small\bfseries, text=orange!70!black] at (4,-0.3) {J4};
    
    % Ligne idéale
    \draw[dashed, blue!60, line width=2pt] (0,5.5) -- (2,2.8) -- (4,0);
    
    % Ligne réelle Sprint 1
    \draw[line width=2.5pt, green!70!black] (0,5.5) -- (1,4) -- (2,2.8);
    
    % Ligne réelle Sprint 2
    \draw[line width=2.5pt, orange!70!black] (2,2.8) -- (3,1.5) -- (4,0);
    
    % Légende avec encadrement
    \node[draw, rounded corners, fill=white, inner sep=5pt] at (3,6.3) {
        \begin{tabular}{ll}
            \textcolor{blue!60}{\rule{1cm}{2pt}} & \small Idéal \\
            \textcolor{green!70!black}{\rule{1cm}{2.5pt}} & \small Sprint 1 (J1-J2) \\
            \textcolor{orange!70!black}{\rule{1cm}{2.5pt}} & \small Sprint 2 (J3-J4)
        \end{tabular}
    };
    
    % Points Sprint 1
    \foreach \x/\y in {0/5.5, 1/4, 2/2.8} {
        \fill[green!70!black] (\x,\y) circle (3pt);
    }
    % Points Sprint 2
    \foreach \x/\y in {2/2.8, 3/1.5, 4/0} {
        \fill[orange!70!black] (\x,\y) circle (3pt);
    }
    
\end{tikzpicture}
\caption{Burndown Chart du Projet FreshMarket}
\end{figure}

\section{Vérification et Validation - Captures d'Écran}

Cette section présente les captures d'écran du projet démontrant l'implémentation des différentes User Stories avec leurs descriptions détaillées.

\subsection{Captures IceScrum - Gestion Agile}

\subsubsection{Product Backlog IceScrum}

\textbf{Description :} Le Product Backlog dans IceScrum contient toutes les User Stories du projet, avec leur priorisation, leurs Story Points et leur assignation aux sprints.

\begin{figure}[H]
\centering
\fbox{\parbox{0.9\textwidth}{
\centering
\vspace{3cm}
\textit{[Capture d'écran : Product Backlog dans IceScrum]}\\
\texttt{screenshots/icescrum\_backlog.png}
\vspace{3cm}
}}
\caption{Product Backlog IceScrum - Vue d'ensemble des User Stories}
\end{figure}

% Pour ajouter la capture : \includegraphics[width=0.9\textwidth]{screenshots/icescrum_backlog.png}

\subsubsection{Sprint Board IceScrum}

\textbf{Description :} Le Sprint Board affiche les tâches en cours, complétées et à faire pour le sprint actif.

\begin{figure}[H]
\centering
\fbox{\parbox{0.9\textwidth}{
\centering
\vspace{3cm}
\textit{[Capture d'écran : Sprint Board dans IceScrum]}\\
\texttt{screenshots/icescrum\_sprint\_board.png}
\vspace{3cm}
}}
\caption{Sprint Board IceScrum - Suivi des tâches}
\end{figure}

% Pour ajouter la capture : \includegraphics[width=0.9\textwidth]{screenshots/icescrum_sprint_board.png}

\subsubsection{Burndown Chart IceScrum}

\textbf{Description :} Le Burndown Chart visualise la progression du sprint avec le travail restant par jour.

\begin{figure}[H]
\centering
\fbox{\parbox{0.9\textwidth}{
\centering
\vspace{3cm}
\textit{[Capture d'écran : Burndown Chart dans IceScrum]}\\
\texttt{screenshots/icescrum\_burndown.png}
\vspace{3cm}
}}
\caption{Burndown Chart IceScrum - Vélocité du sprint}
\end{figure}

% Pour ajouter la capture : \includegraphics[width=0.9\textwidth]{screenshots/icescrum_burndown.png}

\subsection{US-01 : Affichage Liste Produits}

\textbf{Description :} Cette User Story implémente la page d'accueil avec l'affichage de tous les produits disponibles. Les fonctionnalités incluent :
\begin{itemize}
    \item Affichage en grille responsive (Bootstrap)
    \item Pagination des résultats
    \item Informations produit : nom, prix, stock, disponibilité
    \item Bouton "Ajouter au panier" sur chaque carte produit
\end{itemize}

\textbf{Implémentation technique :}
\begin{itemize}
    \item Route : \texttt{/} et \texttt{/products}
    \item Controller : \texttt{HomeController::index()}
    \item Template : \texttt{home/index.html.twig}
    \item Repository : \texttt{ProductRepository::findAllAvailable()}
\end{itemize}

\begin{figure}[H]
\centering
\fbox{\parbox{0.9\textwidth}{
\centering
\vspace{3cm}
\textit{[Capture d'écran : Page d'accueil avec liste des produits en grille]}\\
\texttt{screenshots/us01\_home\_products.png}
\vspace{3cm}
}}
\caption{US-01 - Page d'Accueil avec Liste des Produits Disponibles}
\end{figure}

% Pour ajouter la capture : \includegraphics[width=0.9\textwidth]{screenshots/us01_home_products.png}

\subsection{US-02 : Page Détail Produit}

\textbf{Description :} Page détaillée affichant toutes les informations d'un produit spécifique avec possibilité d'ajout au panier.

\textbf{Fonctionnalités :}
\begin{itemize}
    \item Affichage détaillé : nom, description complète, prix, stock disponible
    \item Sélection de quantité avec validation du stock
    \item Bouton "Ajouter au panier" avec feedback visuel
    \item Indicateur de disponibilité (En stock / Rupture)
\end{itemize}

\textbf{Implémentation technique :}
\begin{itemize}
    \item Route : \texttt{/products/\{id\}}
    \item Controller : \texttt{ProductController::show()}
    \item Template : \texttt{product/show.html.twig}
    \item Validation du stock avant ajout au panier
\end{itemize}

\begin{figure}[H]
\centering
\fbox{\parbox{0.9\textwidth}{
\centering
\vspace{3cm}
\textit{[Capture d'écran : Page détail d'un produit avec informations complètes]}\\
\texttt{screenshots/us02\_product\_detail.png}
\vspace{3cm}
}}
\caption{US-02 - Page Détail Produit avec Bouton Ajouter au Panier}
\end{figure}

% Pour ajouter la capture : \includegraphics[width=0.9\textwidth]{screenshots/us02_product_detail.png}

\subsection{US-03 : Recherche et Filtrage}

\textbf{Description :} Système de recherche et de filtrage permettant aux utilisateurs de trouver rapidement des produits.

\textbf{Fonctionnalités :}
\begin{itemize}
    \item Barre de recherche avec autocomplétion
    \item Filtres par catégorie de produits
    \item Filtre par plage de prix (min/max)
    \item Tri par prix (croissant/décroissant), nom, nouveautés
    \item Résultats paginés avec compteur
\end{itemize}

\textbf{Implémentation technique :}
\begin{itemize}
    \item Route : \texttt{/products/search}
    \item Controller : \texttt{ProductController::search()}
    \item Repository : Requêtes DQL personnalisées avec QueryBuilder
    \item Template : \texttt{product/search.html.twig}
\end{itemize}

\begin{figure}[H]
\centering
\fbox{\parbox{0.9\textwidth}{
\centering
\vspace{3cm}
\textit{[Capture d'écran : Interface de recherche avec filtres actifs]}\\
\texttt{screenshots/us03\_search\_filters.png}
\vspace{3cm}
}}
\caption{US-03 - Recherche et Filtrage des Produits}
\end{figure}

% Pour ajouter la capture : \includegraphics[width=0.9\textwidth]{screenshots/us03_search_filters.png}

\subsection{US-04 : Gestion du Panier}

\textbf{Description :} Système complet de gestion du panier d'achat avec persistance et calculs automatiques.

\textbf{Fonctionnalités :}
\begin{itemize}
    \item Ajout de produits au panier avec sélection de quantité
    \item Modification des quantités directement dans le panier
    \item Suppression d'articles
    \item Calcul automatique du total (avec TVA si applicable)
    \item Persistance du panier (session pour anonymes, DB pour utilisateurs connectés)
    \item Vérification du stock en temps réel
    \item Notifications visuelles (toasts) pour chaque action
\end{itemize}

\textbf{Implémentation technique :}
\begin{itemize}
    \item Service : \texttt{CartService.php} (logique métier réutilisable)
    \item Controller : \texttt{CartController::index()}, \texttt{add()}, \texttt{remove()}, \texttt{update()}
    \item Template : \texttt{cart/index.html.twig}
    \item Stockage : SessionInterface pour les paniers temporaires
\end{itemize}

\begin{figure}[H]
\centering
\fbox{\parbox{0.9\textwidth}{
\centering
\vspace{3cm}
\textit{[Capture d'écran : Page panier avec produits et contrôles de quantité]}\\
\texttt{screenshots/us04\_cart\_management.png}
\vspace{3cm}
}}
\caption{US-04 - Page Panier avec Gestion des Quantités et Calcul du Total}
\end{figure}

% Pour ajouter la capture : \includegraphics[width=0.9\textwidth]{screenshots/us04_cart_management.png}

\subsection{US-05 : Checkout et Création Commande}

\textbf{Description :} Processus complet de finalisation de commande avec validation et confirmation.

\textbf{Fonctionnalités :}
\begin{itemize}
    \item Formulaire d'adresse de livraison avec validation
    \item Récapitulatif de la commande avant validation
    \item Création des entités Order et OrderItem
    \item Mise à jour automatique du stock
    \item Envoi d'email de confirmation
    \item Numéro de commande unique généré
    \item Historique des commandes accessible dans le profil
\end{itemize}

\textbf{Implémentation technique :}
\begin{itemize}
    \item Controller : \texttt{CartController::placeOrder()}
    \item Entities : \texttt{Order.php}, \texttt{OrderItem.php}
    \item Service : \texttt{Symfony Mailer} pour l'envoi d'emails
    \item Template : \texttt{cart/checkout.html.twig}, \texttt{emails/order\_confirmation.html.twig}
    \item Transaction DB pour garantir la cohérence
\end{itemize}

\begin{figure}[H]
\centering
\fbox{\parbox{0.9\textwidth}{
\centering
\vspace{3cm}
\textit{[Capture d'écran : Page de checkout avec formulaire d'adresse]}\\
\texttt{screenshots/us05\_checkout\_form.png}
\vspace{3cm}
}}
\caption{US-05 - Page Checkout avec Formulaire de Livraison}
\end{figure}

% Pour ajouter la capture : \includegraphics[width=0.9\textwidth]{screenshots/us05_checkout_form.png}

\begin{figure}[H]
\centering
\fbox{\parbox{0.9\textwidth}{
\centering
\vspace{3cm}
\textit{[Capture d'écran : Email de confirmation de commande]}\\
\texttt{screenshots/us05\_order\_email.png}
\vspace{3cm}
}}
\caption{US-05 - Email de Confirmation de Commande}
\end{figure}

% Pour ajouter la capture : \includegraphics[width=0.9\textwidth]{screenshots/us05_order_email.png}

\subsection{US-06 : Gestion des Commandes (Admin)}

\textbf{Description :} Interface d'administration pour gérer toutes les commandes des clients.

\textbf{Fonctionnalités :}
\begin{itemize}
    \item Liste de toutes les commandes avec pagination
    \item Filtres par statut (pending, processing, completed, cancelled)
    \item Modification du statut des commandes
    \item Détails complets de chaque commande
    \item Protection CSRF sur toutes les actions
    \item Logs des modifications de statut
    \item Export des commandes (CSV/Excel)
\end{itemize}

\textbf{Implémentation technique :}
\begin{itemize}
    \item Controller : \texttt{Admin/OrderController::index()}, \texttt{show()}, \texttt{updateStatus()}
    \item Template : \texttt{admin/order/index.html.twig}, \texttt{show.html.twig}
    \item Security : \texttt{@IsGranted("ROLE\_ADMIN")}
    \item Repository : Requêtes personnalisées pour statistiques
\end{itemize}

\begin{figure}[H]
\centering
\fbox{\parbox{0.9\textwidth}{
\centering
\vspace{3cm}
\textit{[Capture d'écran : Interface admin de gestion des commandes]}\\
\texttt{screenshots/us06\_admin\_orders.png}
\vspace{3cm}
}}
\caption{US-06 - Interface Administrateur de Gestion des Commandes}
\end{figure}

% Pour ajouter la capture : \includegraphics[width=0.9\textwidth]{screenshots/us06_admin_orders.png}

\subsection{US-07 : Authentification et Rôles}

\textbf{Description :} Système d'authentification complet avec gestion des rôles utilisateur.

\textbf{Fonctionnalités :}
\begin{itemize}
    \item Inscription avec validation de formulaire
    \item Hashage sécurisé des mots de passe (bcrypt)
    \item Connexion avec "Remember me"
    \item Déconnexion sécurisée
    \item Système de rôles : ROLE\_USER, ROLE\_ADMIN
    \item Restriction d'accès par rôle
    \item Protection contre les attaques brute force
\end{itemize}

\textbf{Implémentation technique :}
\begin{itemize}
    \item Controller : \texttt{SecurityController::login()}, \texttt{RegistrationController::register()}
    \item Entity : \texttt{User.php} (implements UserInterface)
    \item Config : \texttt{security.yaml} avec firewalls
    \item Templates : \texttt{security/login.html.twig}, \texttt{registration/register.html.twig}
\end{itemize}

\begin{figure}[H]
\centering
\fbox{\parbox{0.9\textwidth}{
\centering
\vspace{3cm}
\textit{[Capture d'écran : Page de connexion]}\\
\texttt{screenshots/us07\_login\_page.png}
\vspace{3cm}
}}
\caption{US-07 - Page de Connexion}
\end{figure}

% Pour ajouter la capture : \includegraphics[width=0.9\textwidth]{screenshots/us07_login_page.png}

\begin{figure}[H]
\centering
\fbox{\parbox{0.9\textwidth}{
\centering
\vspace{3cm}
\textit{[Capture d'écran : Page d'inscription]}\\
\texttt{screenshots/us07\_register\_page.png}
\vspace{3cm}
}}
\caption{US-07 - Page d'Inscription avec Validation}
\end{figure}

% Pour ajouter la capture : \includegraphics[width=0.9\textwidth]{screenshots/us07_register_page.png}

\subsection{US-08 : Login OAuth (Google/Facebook)}

\textbf{Description :} Intégration de l'authentification OAuth 2.0 avec Google et Facebook.

\textbf{Fonctionnalités :}
\begin{itemize}
    \item Boutons de connexion Google et Facebook
    \item Flux OAuth 2.0 complet
    \item Création automatique de compte si première connexion
    \item Mapping avec comptes existants (par email)
    \item Stockage sécurisé des identifiants OAuth
    \item Tests du flux complet
\end{itemize}

\textbf{Implémentation technique :}
\begin{itemize}
    \item Controllers : \texttt{GoogleOAuthController.php}, \texttt{FacebookOAuthController.php}
    \item Security : \texttt{GoogleAuthenticator.php}, \texttt{FacebookAuthenticator.php}
    \item Bundle : \texttt{knpuniversity/oauth2-client-bundle}
    \item Config : Credentials dans \texttt{.env.local}
    \item Entity User : Champs \texttt{googleId}, \texttt{facebookId}
\end{itemize}

\begin{figure}[H]
\centering
\fbox{\parbox{0.9\textwidth}{
\centering
\vspace{3cm}
\textit{[Capture d'écran : Boutons OAuth sur la page de connexion]}\\
\texttt{screenshots/us08\_oauth\_buttons.png}
\vspace{3cm}
}}
\caption{US-08 - Boutons de Connexion OAuth Google et Facebook}
\end{figure}

% Pour ajouter la capture : \includegraphics[width=0.9\textwidth]{screenshots/us08_oauth_buttons.png}

\begin{figure}[H]
\centering
\fbox{\parbox{0.9\textwidth}{
\centering
\vspace{3cm}
\textit{[Capture d'écran : Flux OAuth Google en action]}\\
\texttt{screenshots/us08\_google\_oauth\_flow.png}
\vspace{3cm}
}}
\caption{US-08 - Flux d'Authentification OAuth Google}
\end{figure}

% Pour ajouter la capture : \includegraphics[width=0.9\textwidth]{screenshots/us08_google_oauth_flow.png}

\subsection{US-09 : Gestion Profil Utilisateur}

\textbf{Description :} Interface permettant aux utilisateurs de gérer leurs informations personnelles.

\textbf{Fonctionnalités :}
\begin{itemize}
    \item Modification du nom et prénom
    \item Changement d'email avec vérification
    \item Upload d'avatar (formats : jpg, png ; taille max : 2MB)
    \item Modification du mot de passe
    \item Historique des commandes
    \item Suppression de compte
\end{itemize}

\textbf{Implémentation technique :}
\begin{itemize}
    \item Controller : \texttt{ProfileController::edit()}, \texttt{uploadAvatar()}
    \item Service : \texttt{FileUploader.php} pour gestion des uploads
    \item Template : \texttt{profile/edit.html.twig}
    \item Validation : Contraintes Symfony (NotBlank, Email, File)
    \item Tests : Tests unitaires pour FileUploader
\end{itemize}

\begin{figure}[H]
\centering
\fbox{\parbox{0.9\textwidth}{
\centering
\vspace{3cm}
\textit{[Capture d'écran : Page de profil utilisateur]}\\
\texttt{screenshots/us09\_profile\_page.png}
\vspace{3cm}
}}
\caption{US-09 - Page de Profil Utilisateur avec Upload Avatar}
\end{figure}

% Pour ajouter la capture : \includegraphics[width=0.9\textwidth]{screenshots/us09_profile_page.png}

\subsection{US-10 : Dashboard Statistiques (Admin)}

\textbf{Description :} Dashboard administrateur avec statistiques et KPIs du site e-commerce.

\textbf{Fonctionnalités :}
\begin{itemize}
    \item Widget chiffre d'affaires total
    \item Graphique évolution CA par mois
    \item Top 10 produits les plus vendus
    \item Liste des commandes récentes
    \item Nombre total d'utilisateurs
    \item Taux de conversion
    \item Graphiques interactifs (Chart.js)
\end{itemize}

\textbf{Implémentation technique :}
\begin{itemize}
    \item Controller : \texttt{Admin/DashboardController::index()}
    \item Repository : Méthodes \texttt{getTotalRevenue()}, \texttt{getTopProducts()}
    \item Template : \texttt{admin/dashboard/index.html.twig}
    \item Security : \texttt{@IsGranted("ROLE\_ADMIN")}
    \item JS : Chart.js pour visualisations
\end{itemize}

\begin{figure}[H]
\centering
\fbox{\parbox{0.9\textwidth}{
\centering
\vspace{3cm}
\textit{[Capture d'écran : Dashboard administrateur avec statistiques]}\\
\texttt{screenshots/us10\_admin\_dashboard.png}
\vspace{3cm}
}}
\caption{US-10 - Dashboard Administrateur avec Statistiques Complètes}
\end{figure}

% Pour ajouter la capture : \includegraphics[width=0.9\textwidth]{screenshots/us10_admin_dashboard.png}

\subsection{US-11 : Tests Automatiques et CI}

\textbf{Description :} Mise en place de tests automatisés et d'intégration continue.

\textbf{Fonctionnalités :}
\begin{itemize}
    \item Tests unitaires (PHPUnit)
    \item Tests fonctionnels (WebTestCase)
    \item Coverage des fonctionnalités critiques
    \item Pipeline CI (GitHub Actions ou similaire)
    \item Tests automatiques à chaque push
    \item Badge de build dans README
\end{itemize}

\textbf{Implémentation technique :}
\begin{itemize}
    \item Framework : PHPUnit (inclus dans Symfony)
    \item Tests : Dossier \texttt{tests/}
    \item Config : \texttt{phpunit.xml.dist}
    \item CI : Fichier \texttt{.github/workflows/ci.yml}
    \item Coverage : PHPUnit avec XDebug
\end{itemize}

\begin{figure}[H]
\centering
\fbox{\parbox{0.9\textwidth}{
\centering
\vspace{3cm}
\textit{[Capture d'écran : Résultats des tests PHPUnit]}\\
\texttt{screenshots/us11\_phpunit\_tests.png}
\vspace{3cm}
}}
\caption{US-11 - Résultats des Tests Automatiques PHPUnit}
\end{figure}

% Pour ajouter la capture : \includegraphics[width=0.9\textwidth]{screenshots/us11_phpunit_tests.png}

\begin{figure}[H]
\centering
\fbox{\parbox{0.9\textwidth}{
\centering
\vspace{3cm}
\textit{[Capture d'écran : Pipeline CI GitHub Actions]}\\
\texttt{screenshots/us11\_ci\_pipeline.png}
\vspace{3cm}
}}
\caption{US-11 - Pipeline d'Intégration Continue}
\end{figure}

% Pour ajouter la capture : \includegraphics[width=0.9\textwidth]{screenshots/us11_ci_pipeline.png}

\subsection{US-12 : Sécurité / Hardening}

\textbf{Description :} Implémentation des mesures de sécurité pour protéger l'application.

\textbf{Fonctionnalités :}
\begin{itemize}
    \item Protection CSRF sur tous les formulaires
    \item Protection XSS (échappement Twig)
    \item Headers de sécurité HTTP (CSP, X-Frame-Options, etc.)
    \item Rate limiting sur les endpoints sensibles
    \item Validation stricte des entrées utilisateur
    \item Tests de sécurité automatisés
\end{itemize}

\textbf{Implémentation technique :}
\begin{itemize}
    \item CSRF : Tokens Symfony intégrés
    \item XSS : Auto-escaping Twig
    \item Headers : Configuration dans \texttt{config/packages/framework.yaml}
    \item Tests : Scripts \texttt{test\_xss.bat}, \texttt{test\_csrf.html}
    \item EventSubscriber : \texttt{SecurityHeadersSubscriber.php}
\end{itemize}

\begin{figure}[H]
\centering
\fbox{\parbox{0.9\textwidth}{
\centering
\vspace{3cm}
\textit{[Capture d'écran : Headers de sécurité HTTP]}\\
\texttt{screenshots/us12\_security\_headers.png}
\vspace{3cm}
}}
\caption{US-12 - Vérification des Headers de Sécurité HTTP}
\end{figure}

% Pour ajouter la capture : \includegraphics[width=0.9\textwidth]{screenshots/us12_security_headers.png}

\begin{figure}[H]
\centering
\fbox{\parbox{0.9\textwidth}{
\centering
\vspace{3cm}
\textit{[Capture d'écran : Test de protection CSRF]}\\
\texttt{screenshots/us12\_csrf\_protection.png}
\vspace{3cm}
}}
\caption{US-12 - Test de Protection CSRF}
\end{figure}

% Pour ajouter la capture : \includegraphics[width=0.9\textwidth]{screenshots/us12_csrf_protection.png}

\subsection{US-13 : Déploiement et Documentation}

\textbf{Description :} Scripts et documentation pour le déploiement de l'application.

\textbf{Livrables :}
\begin{itemize}
    \item Docker Compose fonctionnel
    \item Documentation de déploiement (DEPLOYMENT.md)
    \item Scripts de setup automatisés
    \item Guide de configuration des variables d'environnement
    \item Checklist de vérification post-déploiement
    \item Documentation API
\end{itemize}

\textbf{Fichiers :}
\begin{itemize}
    \item \texttt{docker-compose.yml}
    \item \texttt{Dockerfile}
    \item \texttt{DEPLOYMENT.md}
    \item \texttt{setup\_prod.ps1}
    \item \texttt{.env.example}
\end{itemize}

\begin{figure}[H]
\centering
\fbox{\parbox{0.9\textwidth}{
\centering
\vspace{3cm}
\textit{[Capture d'écran : Application déployée avec Docker]}\\
\texttt{screenshots/us13\_docker\_deployment.png}
\vspace{3cm}
}}
\caption{US-13 - Application Déployée avec Docker Compose}
\end{figure}

% Pour ajouter la capture : \includegraphics[width=0.9\textwidth]{screenshots/us13_docker_deployment.png}

\section{Outils de Gestion de Projet}

Le projet a été géré avec les outils suivants :

\begin{table}[H]
\centering
\caption{Outils Agile Utilisés}
\begin{tabularx}{\textwidth}{|l|X|}
\hline
\textbf{Outil} & \textbf{Usage} \\
\hline
\hline
\textbf{IceScrum} & Gestion du Product Backlog, planification des Sprints \\
\hline
\textbf{Git / GitHub} & Versioning du code, branches par feature \\
\hline
\textbf{Composer} & Gestion des dépendances PHP \\
\hline
\textbf{Symfony CLI} & Développement et débogage local \\
\hline
\textbf{PHPUnit} & Tests automatisés (unitaires et fonctionnels) \\
\hline
\end{tabularx}
\end{table}

% ============================================================================
%                    CHAPITRE 3 : ARCHITECTURE DU PROJET
% ============================================================================
\chapter{Architecture du Projet}

\section{Structure MVC de Symfony}

Symfony impose une architecture \textbf{MVC} (Model-View-Controller) stricte. Le projet respecte cette séparation :

\begin{figure}[H]
\centering
\begin{tikzpicture}[
    node distance=2.5cm,
    box/.style={rectangle, draw=blue!70!black, line width=2pt, rounded corners=10pt, minimum width=4.5cm, minimum height=1.5cm, align=center, fill=blue!15, drop shadow, font=\bfseries\large},
    arrow/.style={-{Stealth[length=3mm]}, line width=1.5pt, blue!60!black},
    label/.style={font=\small, text=black!70}
]
    % Nodes
    \node[box, fill=cyan!20] (browser) {Navigateur\\(Client)};
    \node[box, below=of browser, fill=green!20] (controller) {Controller};
    \node[box, left=3.5cm of controller, fill=orange!20] (model) {Model\\(Entity)};
    \node[box, right=3.5cm of controller, fill=purple!20] (view) {View\\(Twig)};
    \node[box, below=of model, fill=yellow!25] (repository) {Repository};
    \node[box, below=of repository, fill=red!20] (database) {Base de\\Données};
    
    % Arrows
    \draw[arrow] (browser) -- node[right, label] {HTTP Request} (controller);
    \draw[arrow] (controller) -- node[above, label] {Fetch Data} (model);
    \draw[arrow] (model) -- node[right, label] {Query} (repository);
    \draw[arrow] (repository) -- node[right, label] {SQL} (database);
    \draw[arrow, dashed] (database) -- node[left, label] {Results} (repository);
    \draw[arrow, dashed] (repository) -- node[left, label] {Entity} (model);
    \draw[arrow, dashed] (model) -- node[below, label] {Data} (controller);
    \draw[arrow] (controller) -- node[above, label] {Variables} (view);
    \draw[arrow, dashed] (view) -- node[left, label] {HTML Response} (browser);
    
\end{tikzpicture}
\caption{Flux MVC dans Symfony}
\end{figure}

\section{Arborescence des Fichiers}

\begin{lstlisting}[style=yamlstyle, caption={Structure du projet}]
synf_project/
|-- src/
|   |-- Controller/        # Logique HTTP (routes)
|   |   |-- CartController.php
|   |   |-- HomeController.php
|   |   |-- ProductController.php
|   |   |-- ProfileController.php
|   |   |-- SecurityController.php
|   |   |-- GoogleOAuthController.php
|   |   |-- FacebookOAuthController.php
|   |   |-- Admin/         # Controleurs administration
|   |-- Entity/            # Modeles Doctrine (tables BDD)
|   |   |-- User.php
|   |   |-- Product.php
|   |   |-- Order.php
|   |   |-- OrderItem.php
|   |   |-- Reservation.php
|   |-- Repository/        # Requetes SQL personnalisees
|   |-- Service/           # Logique metier reutilisable
|   |   |-- CartService.php
|   |   |-- FileUploader.php
|   |-- Form/              # Formulaires Symfony
|   |-- Security/          # Authenticators OAuth
|   |-- EventSubscriber/   # Listeners d'evenements
|-- templates/             # Fichiers Twig (Vue)
|-- config/                # Configuration YAML
|-- public/                # Point d'entree web
|-- vendor/                # Dependances Composer
\end{lstlisting}

% ============================================================================
%                    CHAPITRE 3 : ENTITÉS DOCTRINE (MODÈLE)
% ============================================================================
\chapter{Entités Doctrine - Le Modèle de Données}

\section{Vue d'Ensemble des Entités}

Le modèle de données est composé de 5 entités principales interconnectées :

\begin{figure}[H]
\centering
\begin{tikzpicture}[
    entity/.style={
        rectangle, draw=doctrinecolor, line width=2.5pt, 
        rounded corners=10pt, minimum width=4.5cm, minimum height=1.5cm,
        fill=orange!20, font=\bfseries\ttfamily\Large, drop shadow, align=center
    },
    relation/.style={-{Stealth[length=3mm]}, line width=1.5pt, gray!70!black},
    label/.style={font=\small\bfseries, text=doctrinecolor}
]
    \node[entity] (user) at (0,0) {User};
    \node[entity] (order) at (-5,-3.5) {Order};
    \node[entity] (reservation) at (5,-3.5) {Reservation};
    \node[entity] (orderitem) at (-5,-7) {OrderItem};
    \node[entity] (product) at (0,-7) {Product};
    
    % Relations avec labels améliorés
    \draw[relation] (user) -- node[left, label] {1} node[left, label, below=2mm] {*} (order);
    \draw[relation] (user) -- node[right, label] {1} node[right, label, below=2mm] {*} (reservation);
    \draw[relation] (order) -- node[left, label] {1} node[left, label, below=2mm] {*} (orderitem);
    \draw[relation] (orderitem) -- node[above, label] {*} node[below, label, xshift=-5mm] {1} (product);
\end{tikzpicture}
\caption{Diagramme des Relations entre Entités}
\end{figure}

\section{Entité \texttt{Product} - Étude de Cas Détaillée}

L'entité \entity{Product} représente un article du catalogue. Voici une analyse \textbf{ligne par ligne} de son code source :

\subsection{Déclaration et Attributs ORM}

\begin{lstlisting}[style=phpstyle, caption={Product.php - Déclaration}]
<?php

namespace App\Entity;

use App\Repository\ProductRepository;
use Doctrine\Common\Collections\ArrayCollection;
use Doctrine\Common\Collections\Collection;
use Doctrine\DBAL\Types\Types;
use Doctrine\ORM\Mapping as ORM;
use Symfony\Component\Validator\Constraints as Assert;

#[ORM\Entity(repositoryClass: ProductRepository::class)]
class Product
{
\end{lstlisting}

\textbf{Explications :}
\begin{itemize}
    \item \textbf{Ligne 3-10 :} Imports des classes nécessaires. \code{ORM} contient les annotations Doctrine, \code{Assert} les contraintes de validation.
    \item \textbf{Ligne 12 :} L'attribut PHP 8 \code{\#[ORM\textbackslash Entity]} indique à Doctrine que cette classe est une entité persistée en base. Le paramètre \code{repositoryClass} lie cette entité à son Repository personnalisé.
\end{itemize}

\subsection{Propriétés et Colonnes}

\begin{lstlisting}[style=phpstyle, caption={Product.php - Propriétés}]
    #[ORM\Id]
    #[ORM\GeneratedValue]
    #[ORM\Column]
    private ?int $id = null;

    #[ORM\Column(length: 255)]
    #[Assert\NotBlank(message: 'Le nom est obligatoire')]
    private ?string $nom = null;

    #[ORM\Column(type: Types::DECIMAL, precision: 10, scale: 2)]
    #[Assert\NotBlank(message: 'Le prix est obligatoire')]
    #[Assert\Positive(message: 'Le prix doit etre positif')]
    private ?string $prix = null;

    #[ORM\Column]
    private ?int $stock = 0;

    #[ORM\Column]
    private ?bool $disponible = true;
\end{lstlisting}

\begin{table}[H]
\centering
\caption{Mapping des Propriétés vers la Base de Données}
\begin{tabularx}{\textwidth}{|l|l|X|}
\hline
\textbf{Propriété PHP} & \textbf{Type SQL} & \textbf{Description} \\
\hline
\hline
\code{\$id} & INT AUTO\_INCREMENT & Clé primaire générée automatiquement \\
\hline
\code{\$nom} & VARCHAR(255) & Nom du produit, ne peut pas être vide \\
\hline
\code{\$prix} & DECIMAL(10,2) & Prix en euros avec 2 décimales \\
\hline
\code{\$stock} & INT & Quantité disponible en stock \\
\hline
\code{\$disponible} & TINYINT(1) & Booléen indiquant si le produit est actif \\
\hline
\end{tabularx}
\end{table}

\subsection{Relations avec OrderItem}

\begin{lstlisting}[style=phpstyle, caption={Product.php - Relation OneToMany}]
    #[ORM\OneToMany(mappedBy: 'product', targetEntity: OrderItem::class)]
    private Collection $orderItems;

    public function __construct()
    {
        $this->createdAt = new \DateTime();
        $this->orderItems = new ArrayCollection();
    }
\end{lstlisting}

\textbf{Explication de la relation :}
\begin{itemize}
    \item \code{OneToMany} : Un produit peut apparaître dans plusieurs lignes de commande.
    \item \code{mappedBy: 'product'} : La clé étrangère est définie du côté \entity{OrderItem} dans sa propriété \code{\$product}.
    \item \code{ArrayCollection} : Collection Doctrine optimisée pour le lazy loading.
\end{itemize}

\subsection{Getters et Setters - Pattern Fluent}

\begin{lstlisting}[style=phpstyle, caption={Product.php - Méthodes Fluent}]
    public function getNom(): ?string
    {
        return $this->nom;
    }

    public function setNom(string $nom): static
    {
        $this->nom = $nom;
        return $this;  // Pattern Fluent Interface
    }
\end{lstlisting}

Le pattern \textbf{Fluent Interface} permet d'enchaîner les appels :
\begin{lstlisting}[style=phpstyle]
$product = (new Product())
    ->setNom('Pommes Bio')
    ->setPrix('2.99')
    ->setStock(100)
    ->setDisponible(true);
\end{lstlisting}

\section{Entité \texttt{User} - Authentification}

L'entité \entity{User} implémente les interfaces de sécurité Symfony :

\begin{lstlisting}[style=phpstyle, caption={User.php - Interfaces de Sécurité}]
class User implements UserInterface, PasswordAuthenticatedUserInterface
{
    #[ORM\Column(length: 180, unique: true)]
    private ?string $email = null;

    #[ORM\Column]
    private array $roles = [];

    #[ORM\Column]
    private ?string $password = null;

    #[ORM\Column(length: 255, nullable: true, unique: true)]
    private ?string $googleId = null;

    #[ORM\Column(length: 255, nullable: true, unique: true)]
    private ?string $facebookId = null;
\end{lstlisting}

\textbf{Points clés :}
\begin{description}
    \item[\code{UserInterface}] Requiert les méthodes \code{getUserIdentifier()}, \code{getRoles()}, \code{eraseCredentials()}.
    \item[\code{PasswordAuthenticatedUserInterface}] Requiert \code{getPassword()}.
    \item[\code{googleId} / \code{facebookId}] Permettent l'association OAuth sans doublon.
\end{description}

\section{Diagramme de Classes UML Complet (TikZ)}

\begin{figure}[H]
\centering
\resizebox{\textwidth}{!}{%
\begin{tikzpicture}[
    umlclass/.style={
        rectangle split, rectangle split parts=3,
        draw=blue!70!black, line width=1.5pt,
        rounded corners=5pt,
        minimum width=5cm,
        fill=blue!10,
        font=\small\ttfamily,
        drop shadow
    },
    umlarrow/.style={-{Triangle[open, length=3mm]}, line width=1.5pt},
    umlcomp/.style={-{Diamond[fill=black, length=3mm]}, line width=1.5pt}
]

    % User Class
    \node[umlclass] (User) at (0,0) {
        \textbf{User}
        \nodepart{second}
        - id : int\\
        - email : string\\
        - password : string\\
        - roles : array\\
        - nom : string\\
        - prenom : string\\
        - googleId : string\\
        - facebookId : string\\
        - avatar : string
        \nodepart{third}
        + getFullName() : string\\
        + getRoles() : array\\
        + getUserIdentifier() : string
    };

    % Product Class
    \node[umlclass] (Product) at (8,0) {
        \textbf{Product}
        \nodepart{second}
        - id : int\\
        - nom : string\\
        - prix : decimal\\
        - stock : int\\
        - categorie : string\\
        - disponible : bool\\
        - image : string
        \nodepart{third}
        + getNom() : string\\
        + getPrix() : string\\
        + getStock() : int
    };

    % Order Class
    \node[umlclass] (Order) at (0,-8) {
        \textbf{Order}
        \nodepart{second}
        - id : int\\
        - numeroCommande : string\\
        - montantTotal : decimal\\
        - statut : string\\
        - adresseLivraison : text\\
        - createdAt : datetime
        \nodepart{third}
        + calculateTotal() : void\\
        + getStatutLabel() : string\\
        + getOrderItems() : Collection
    };

    % OrderItem Class
    \node[umlclass] (OrderItem) at (8,-8) {
        \textbf{OrderItem}
        \nodepart{second}
        - id : int\\
        - quantite : int\\
        - prixUnitaire : decimal
        \nodepart{third}
        + getTotal() : float\\
        + getProduct() : Product\\
        + getCommande() : Order
    };

    % Reservation Class
    \node[umlclass] (Reservation) at (16,0) {
        \textbf{Reservation}
        \nodepart{second}
        - id : int\\
        - dateReservation : date\\
        - heureReservation : time\\
        - nombrePersonnes : int\\
        - statut : string\\
        - commentaire : text
        \nodepart{third}
        + getStatutLabel() : string\\
        + getUser() : User
    };

    % Relations
    \draw[umlarrow] (Order.north) -- ++(0,1) -| node[pos=0.25, above] {*..1} (User.south);
    \draw[umlarrow] (Reservation.west) -- node[above] {*..1} (User.east);
    \draw[umlcomp] (Order.east) -- node[above] {1..*} (OrderItem.west);
    \draw[umlarrow] (OrderItem.north) -- ++(0,1) -| node[pos=0.25, above] {*..1} (Product.south);

\end{tikzpicture}
}
\caption{Diagramme de Classes UML Complet}
\end{figure}

% ============================================================================
%                    CHAPITRE 4 : CONTRÔLEURS
% ============================================================================
\chapter{Contrôleurs - La Logique HTTP}

\section{Rôle des Contrôleurs dans Symfony}

Un contrôleur est une classe PHP dont les méthodes (appelées \textbf{actions}) répondent aux requêtes HTTP. Chaque action :
\begin{enumerate}
    \item Reçoit un objet \code{Request}
    \item Traite la logique métier (souvent via des Services)
    \item Retourne un objet \code{Response} (HTML, JSON, redirection)
\end{enumerate}

\section{\texttt{CartController} - Gestion du Panier}

Le \controller{CartController} gère toutes les opérations du panier d'achat.

\subsection{Structure et Attributs de Classe}

\begin{lstlisting}[style=phpstyle, caption={CartController.php - Déclaration}]
<?php

namespace App\Controller;

use App\Entity\Product;
use App\Entity\Order;
use App\Entity\OrderItem;
use App\Repository\ProductRepository;
use App\Service\CartService;
use Doctrine\ORM\EntityManagerInterface;
use Symfony\Bundle\FrameworkBundle\Controller\AbstractController;
use Symfony\Component\HttpFoundation\Request;
use Symfony\Component\HttpFoundation\Response;
use Symfony\Component\Routing\Attribute\Route;
use Symfony\Component\Security\Http\Attribute\IsGranted;

#[Route('/cart')]
class CartController extends AbstractController
{
\end{lstlisting}

\textbf{Explications :}
\begin{itemize}
    \item \code{\#[Route('/cart')]} : Préfixe de route appliqué à toutes les actions de ce contrôleur.
    \item \code{AbstractController} : Classe parente fournissant des helpers (\code{render()}, \code{redirectToRoute()}, \code{json()}).
\end{itemize}

\subsection{Action \texttt{add()} - Ajout au Panier}

\begin{lstlisting}[style=phpstyle, caption={CartController::add()}]
#[Route('/add/{id}', name: 'app_cart_add', requirements: ['id' => '\d+'])]
#[IsGranted('ROLE_USER')]
public function add(int $id, ProductRepository $productRepository, 
                    CartService $cartService): Response
{
    $product = $productRepository->find($id);
    
    if (!$product) {
        $this->addFlash('error', 'Produit introuvable.');
        return $this->redirectToRoute('app_product_index');
    }
    
    if ($product->getStock() <= 0) {
        $this->addFlash('error', 'Ce produit est en rupture de stock.');
        return $this->redirectToRoute('app_product_show', ['id' => $id]);
    }
    
    // Ajouter le produit au panier (stockage optimise: ID seulement)
    $cartService->addProduct($id);
    
    $this->addFlash('success', 'Produit ajoute au panier avec succes !');
    
    return $this->redirectToRoute('app_product_show', ['id' => $id]);
}
\end{lstlisting}

\begin{table}[H]
\centering
\caption{Analyse de l'action \texttt{add()}}
\begin{tabularx}{\textwidth}{|l|X|}
\hline
\textbf{Élément} & \textbf{Explication} \\
\hline
\hline
\code{requirements: ['id' => '\textbackslash d+']} & Validation regex : l'ID doit être numérique \\
\hline
\code{\#[IsGranted('ROLE\_USER')]} & Seuls les utilisateurs connectés accèdent à cette route \\
\hline
\code{ProductRepository \$productRepository} & Injection automatique de dépendance (autowiring) \\
\hline
\code{addFlash()} & Message flash stocké en session, affiché une seule fois \\
\hline
\code{redirectToRoute()} & Redirection HTTP 302 vers une route nommée \\
\hline
\end{tabularx}
\end{table}

\subsection{Action \texttt{placeOrder()} - Validation de Commande}

\begin{lstlisting}[style=phpstyle, caption={CartController::placeOrder()}]
#[Route('/place-order', name: 'app_cart_place_order', methods: ['POST'])]
#[IsGranted('ROLE_USER')]
public function placeOrder(Request $request, CartService $cartService, 
                           EntityManagerInterface $em): Response
{
    // Verification du token CSRF
    $token = $request->request->get('_token');
    if (!$this->isCsrfTokenValid('place-order', $token)) {
        $this->addFlash('error', 'Erreur de securite. Veuillez reessayer.');
        return $this->redirectToRoute('app_cart_checkout');
    }
    
    // Creer la commande
    $order = new Order();
    $order->setUser($this->getUser());
    $order->setStatut('en_attente');
    
    // Creer les OrderItems
    $total = 0;
    foreach ($cartService->getCart() as $productId => $item) {
        $product = $item['product'];
        
        $orderItem = new OrderItem();
        $orderItem->setProduct($product);
        $orderItem->setQuantite($item['quantity']);
        $orderItem->setPrixUnitaire($product->getPrix());
        
        $order->addOrderItem($orderItem);
        
        $total += $product->getPrix() * $item['quantity'];
    }
    
    $order->setMontantTotal((string)$total);
    
    // Persister (cascade sur OrderItems)
    $em->persist($order);
    $em->flush();
    
    // Vider le panier
    $cartService->clear();
    
    return $this->redirectToRoute('app_profile_orders');
}
\end{lstlisting}

\section{Diagramme de Séquence - Processus de Commande}

\begin{figure}[H]
\centering
\begin{tikzpicture}[
    actor/.style={rectangle, draw=blue!70!black, line width=1.5pt, rounded corners=5pt, minimum width=2.5cm, minimum height=1cm, fill=blue!15, font=\bfseries\small, drop shadow},
    lifeline/.style={dashed, gray!60, line width=1pt},
    message/.style={-{Stealth[length=3mm]}, line width=1.5pt, blue!70!black},
    returnmsg/.style={-{Stealth[length=3mm]}, dashed, line width=1.5pt, green!70!black}
]
    % Actors
    \node[actor] (user) at (0,0) {Utilisateur};
    \node[actor] (ctrl) at (4,0) {CartController};
    \node[actor] (srv) at (8,0) {CartService};
    \node[actor] (db) at (12,0) {Database};
    
    % Lifelines
    \draw[lifeline] (user.south) -- ++(0,-10);
    \draw[lifeline] (ctrl.south) -- ++(0,-10);
    \draw[lifeline] (srv.south) -- ++(0,-10);
    \draw[lifeline] (db.south) -- ++(0,-10);
    
    % Messages
    \draw[message] (0,-1) -- node[above, font=\scriptsize] {1. POST /place-order} (4,-1);
    \draw[message] (4,-2) -- node[above, font=\scriptsize] {2. isEmpty()} (8,-2);
    \draw[returnmsg] (8,-2.5) -- node[above, font=\scriptsize] {false} (4,-2.5);
    \draw[message] (4,-3.5) -- node[above, font=\scriptsize] {3. getCart()} (8,-3.5);
    \draw[returnmsg] (8,-4) -- node[above, font=\scriptsize] {array} (4,-4);
    \draw[message] (4,-5) -- node[above, font=\scriptsize] {4. persist(Order)} (12,-5);
    \draw[message] (4,-6) -- node[above, font=\scriptsize] {5. flush()} (12,-6);
    \draw[returnmsg] (12,-6.5) -- node[above, font=\scriptsize] {OK} (4,-6.5);
    \draw[message] (4,-7.5) -- node[above, font=\scriptsize] {6. clear()} (8,-7.5);
    \draw[returnmsg] (4,-9) -- node[above, font=\scriptsize] {7. Redirect /profile/orders} (0,-9);
    
\end{tikzpicture}
\caption{Diagramme de Séquence - Validation de Commande}
\end{figure}

% ============================================================================
%                    CHAPITRE 5 : SERVICES
% ============================================================================
\chapter{Services - La Logique Métier}

\section{Architecture des Services}

Les \textbf{Services} dans Symfony sont des classes PHP réutilisables contenant la logique métier. Ils sont automatiquement injectés grâce à l'\textbf{autowiring}.

\section{\texttt{CartService} - Gestion du Panier en Cookie}

Le \service{CartService} gère le panier côté client via des cookies encodés en Base64.

\begin{lstlisting}[style=phpstyle, caption={CartService.php - Propriétés}]
<?php

namespace App\Service;

use App\Repository\ProductRepository;
use Symfony\Component\HttpFoundation\RequestStack;
use Symfony\Component\HttpFoundation\Cookie;

class CartService
{
    private const CART_COOKIE_NAME = 'freshmarket_cart_token';
    private const COOKIE_LIFETIME = 60 * 60 * 24 * 30; // 30 jours
    
    private array $cart = [];

    public function __construct(
        private RequestStack $requestStack,
        private ProductRepository $productRepository
    ) {
        $this->loadCartFromCookie();
    }
\end{lstlisting}

\subsection{Méthodes Détaillées}

\begin{table}[H]
\centering
\caption{Méthodes du CartService}
\begin{tabularx}{\textwidth}{|l|l|X|}
\hline
\textbf{Méthode} & \textbf{Retour} & \textbf{Description} \\
\hline
\hline
\code{addProduct(\$id, \$qty)} & bool & Ajoute un produit au panier \\
\hline
\code{removeProduct(\$id)} & void & Supprime un produit du panier \\
\hline
\code{updateQuantity(\$id, \$qty)} & void & Met à jour la quantité \\
\hline
\code{getCart()} & array & Retourne le panier avec objets Product \\
\hline
\code{getRawCart()} & array & Retourne le panier brut (IDs) \\
\hline
\code{getTotal()} & float & Calcule le montant total \\
\hline
\code{getCount()} & int & Nombre de produits différents \\
\hline
\code{isEmpty()} & bool & Vérifie si le panier est vide \\
\hline
\code{clear()} & void & Vide le panier \\
\hline
\code{createCookie()} & Cookie & Génère le cookie à envoyer \\
\hline
\end{tabularx}
\end{table}

\begin{lstlisting}[style=phpstyle, caption={CartService.php - Méthode getCart()}]
public function getCart(): array
{
    $cartWithProducts = [];
    foreach ($this->cart as $productId => $quantity) {
        $product = $this->productRepository->find($productId);
        
        if ($product && $product->getStock() > 0) {
            $cartWithProducts[$productId] = [
                'product' => $product,
                'quantity' => $quantity
            ];
        } else {
            // Produit inexistant ou rupture : on le retire
            $this->removeProduct($productId);
        }
    }
    
    return $cartWithProducts;
}
\end{lstlisting}

\textbf{Avantages de cette architecture :}
\begin{itemize}
    \item Le panier persiste 30 jours même sans connexion.
    \item Les données en cookie sont légères (uniquement IDs).
    \item La reconstruction est faite à la demande (lazy loading).
\end{itemize}

% ============================================================================
%                    CHAPITRE 6 : SÉCURITÉ
% ============================================================================
\chapter{Sécurité - Protection de l'Application}

\section{Configuration du Pare-feu Symfony}

Le fichier \texttt{config/packages/security.yaml} définit toute la stratégie de sécurité :

\begin{lstlisting}[style=yamlstyle, caption={security.yaml}]
security:
    password_hashers:
        Symfony\Component\Security\Core\User\PasswordAuthenticatedUserInterface: 'auto'
        App\Entity\User:
            algorithm: auto  # bcrypt ou argon2 selon le serveur
    
    providers:
        app_user_provider:
            entity:
                class: App\Entity\User
                property: email  # Identifiant de connexion
    
    firewalls:
        main:
            lazy: true
            provider: app_user_provider
            custom_authenticators:
                - App\Security\GoogleAuthenticator
                - App\Security\FacebookAuthenticator
            form_login:
                login_path: app_login
                check_path: app_login
                enable_csrf: true
                username_parameter: email
                password_parameter: password
            logout:
                path: app_logout
                target: app_home
            remember_me:
                secret: '%kernel.secret%'
                lifetime: 604800  # 7 jours
    
    access_control:
        - { path: ^/admin, roles: ROLE_ADMIN }
        - { path: ^/profile, roles: ROLE_USER }
        - { path: ^/cart, roles: ROLE_USER }
\end{lstlisting}

\section{OAuth 2.0 - Authentification Sociale}

\subsection{GoogleAuthenticator - Flux Complet}

\begin{lstlisting}[style=phpstyle, caption={GoogleAuthenticator.php - authenticate()}]
public function authenticate(Request $request): Passport
{
    $client = $this->clientRegistry->getClient('google');
    $accessToken = $this->fetchAccessToken($client);
    
    /** @var GoogleUser $googleUser */
    $googleUser = $client->fetchUserFromToken($accessToken);

    return new SelfValidatingPassport(
        new UserBadge($googleUser->getEmail(), function () use ($googleUser) {
            return $this->getOrCreateUser($googleUser);
        }),
        [new RememberMeBadge()]
    );
}
\end{lstlisting}

\begin{lstlisting}[style=phpstyle, caption={GoogleAuthenticator.php - getOrCreateUser()}]
private function getOrCreateUser(GoogleUser $googleUser): User
{
    $googleId = $googleUser->getId();
    $email = $googleUser->getEmail();

    // 1. Chercher par Google ID
    $existingUser = $this->userRepository->findOneBy(['googleId' => $googleId]);
    if ($existingUser instanceof User) {
        return $existingUser;
    }

    // 2. Chercher par email (lier compte existant)
    $userByEmail = $this->userRepository->findOneBy(['email' => $email]);
    if ($userByEmail instanceof User) {
        $userByEmail->setGoogleId($googleId);
        $this->entityManager->flush();
        return $userByEmail;
    }

    // 3. Creer un nouvel utilisateur
    $user = new User();
    $user->setEmail($email);
    $user->setGoogleId($googleId);
    $user->setPrenom($googleUser->getFirstName() ?: 'Utilisateur');
    $user->setNom($googleUser->getLastName() ?: 'Google');
    
    // Mot de passe aleatoire (non utilisable directement)
    $randomPassword = bin2hex(random_bytes(16));
    $user->setPassword($this->passwordHasher->hashPassword($user, $randomPassword));

    $this->entityManager->persist($user);
    $this->entityManager->flush();

    return $user;
}
\end{lstlisting}

\section{Headers de Sécurité HTTP}

Le \code{SecurityHeadersSubscriber} injecte des headers de protection :

\begin{lstlisting}[style=phpstyle, caption={SecurityHeadersSubscriber.php}]
public function onKernelResponse(ResponseEvent $event): void
{
    $response = $event->getResponse();
    $headers = $response->headers;

    // Protection XSS
    $headers->set('X-XSS-Protection', '1; mode=block');

    // Protection Clickjacking
    $headers->set('X-Frame-Options', 'DENY');

    // Protection MIME Sniffing
    $headers->set('X-Content-Type-Options', 'nosniff');

    // Referrer Policy
    $headers->set('Referrer-Policy', 'strict-origin-when-cross-origin');

    // Content Security Policy
    $csp = implode('; ', [
        "default-src 'self'",
        "script-src 'self' 'unsafe-inline' https://cdn.jsdelivr.net",
        "style-src 'self' 'unsafe-inline' https://cdn.jsdelivr.net",
        "img-src 'self' data: https:",
        "frame-ancestors 'none'",
    ]);
    $headers->set('Content-Security-Policy', $csp);
}
\end{lstlisting}

\begin{table}[H]
\centering
\caption{Headers de Sécurité et leurs Protections}
\begin{tabularx}{\textwidth}{|l|X|}
\hline
\textbf{Header} & \textbf{Protection Fournie} \\
\hline
\hline
\code{X-XSS-Protection} & Active le filtre XSS du navigateur \\
\hline
\code{X-Frame-Options: DENY} & Empêche l'inclusion dans une iframe (Clickjacking) \\
\hline
\code{X-Content-Type-Options} & Empêche le MIME sniffing \\
\hline
\code{Content-Security-Policy} & Contrôle les sources de scripts/styles autorisées \\
\hline
\code{Referrer-Policy} & Limite les informations envoyées à des sites tiers \\
\hline
\code{Strict-Transport-Security} & Force HTTPS (production uniquement) \\
\hline
\end{tabularx}
\end{table}

\section{Protection CSRF}

Symfony active la protection CSRF par défaut sur les formulaires :

\begin{lstlisting}[style=twigstyle, caption={Template Twig avec CSRF}]
<form method="post" action="{{ path('app_cart_place_order') }}">
    <input type="hidden" name="_token" 
           value="{{ csrf_token('place-order') }}">
    <!-- ... champs du formulaire ... -->
    <button type="submit">Valider la commande</button>
</form>
\end{lstlisting}

\begin{lstlisting}[style=phpstyle, caption={Validation CSRF côté contrôleur}]
$token = $request->request->get('_token');
if (!$this->isCsrfTokenValid('place-order', $token)) {
    throw new AccessDeniedException('Token CSRF invalide');
}
\end{lstlisting}

% ============================================================================
%                    CHAPITRE 7 : REPOSITORIES
% ============================================================================
\chapter{Repositories - Les Requêtes Personnalisées}

\section{Rôle des Repositories}

Les \textbf{Repositories} encapsulent les requêtes SQL complexes. Ils héritent de \code{ServiceEntityRepository} fourni par Doctrine.

\section{\texttt{ProductRepository} - Exemple Complet}

\begin{lstlisting}[style=phpstyle, caption={ProductRepository.php}]
<?php

namespace App\Repository;

use App\Entity\Product;
use Doctrine\Bundle\DoctrineBundle\Repository\ServiceEntityRepository;
use Doctrine\Persistence\ManagerRegistry;

class ProductRepository extends ServiceEntityRepository
{
    public function __construct(ManagerRegistry $registry)
    {
        parent::__construct($registry, Product::class);
    }

    /**
     * Trouve tous les produits disponibles et en stock
     */
    public function findAvailable(): array
    {
        return $this->createQueryBuilder('p')
            ->andWhere('p.disponible = :disponible')
            ->andWhere('p.stock > 0')
            ->setParameter('disponible', true)
            ->orderBy('p.nom', 'ASC')
            ->getQuery()
            ->getResult();
    }

    /**
     * Recherche par categorie
     */
    public function findByCategory(string $category): array
    {
        return $this->createQueryBuilder('p')
            ->andWhere('p.categorie = :category')
            ->andWhere('p.disponible = :disponible')
            ->setParameter('category', $category)
            ->setParameter('disponible', true)
            ->orderBy('p.nom', 'ASC')
            ->getQuery()
            ->getResult();
    }

    /**
     * Recherche textuelle dans nom et description
     */
    public function searchProducts(string $query): array
    {
        return $this->createQueryBuilder('p')
            ->andWhere('p.nom LIKE :query OR p.description LIKE :query')
            ->andWhere('p.disponible = :disponible')
            ->setParameter('query', '%'.$query.'%')
            ->setParameter('disponible', true)
            ->orderBy('p.nom', 'ASC')
            ->getQuery()
            ->getResult();
    }
}
\end{lstlisting}

\textbf{Avantages du QueryBuilder :}
\begin{itemize}
    \item Requêtes sécurisées (préparation automatique des paramètres).
    \item Code lisible et maintenable.
    \item Génération SQL optimisée par Doctrine.
\end{itemize}

% ============================================================================
%                    CHAPITRE 8 : FORMULAIRES
% ============================================================================
\chapter{Formulaires Symfony}

\section{Architecture des Formulaires}

Symfony utilise le pattern \textbf{Form Type} pour définir des formulaires réutilisables.

\section{\texttt{RegistrationFormType} - Inscription}

\begin{lstlisting}[style=phpstyle, caption={RegistrationFormType.php}]
<?php

namespace App\Form;

use App\Entity\User;
use Symfony\Component\Form\AbstractType;
use Symfony\Component\Form\Extension\Core\Type\CheckboxType;
use Symfony\Component\Form\Extension\Core\Type\EmailType;
use Symfony\Component\Form\Extension\Core\Type\PasswordType;
use Symfony\Component\Form\Extension\Core\Type\RepeatedType;
use Symfony\Component\Form\FormBuilderInterface;
use Symfony\Component\OptionsResolver\OptionsResolver;
use Symfony\Component\Validator\Constraints\IsTrue;
use Symfony\Component\Validator\Constraints\Length;
use Symfony\Component\Validator\Constraints\NotBlank;

class RegistrationFormType extends AbstractType
{
    public function buildForm(FormBuilderInterface $builder, array $options): void
    {
        $builder
            ->add('email', EmailType::class, [
                'label' => 'Email',
                'attr' => [
                    'class' => 'form-control',
                    'placeholder' => 'votre@email.com'
                ]
            ])
            ->add('nom', TextType::class, [
                'label' => 'Nom',
                'attr' => ['class' => 'form-control']
            ])
            ->add('prenom', TextType::class, [
                'label' => 'Prenom',
                'attr' => ['class' => 'form-control']
            ])
            ->add('agreeTerms', CheckboxType::class, [
                'label' => 'J\'accepte les conditions',
                'mapped' => false,  // Non lie a l'entite
                'constraints' => [
                    new IsTrue([
                        'message' => 'Vous devez accepter les conditions.',
                    ]),
                ],
            ])
            ->add('plainPassword', RepeatedType::class, [
                'type' => PasswordType::class,
                'mapped' => false,
                'first_options' => [
                    'label' => 'Mot de passe',
                    'constraints' => [
                        new NotBlank(['message' => 'Mot de passe requis']),
                        new Length([
                            'min' => 6,
                            'minMessage' => 'Minimum {{ limit }} caracteres',
                        ]),
                    ],
                ],
                'second_options' => [
                    'label' => 'Confirmer le mot de passe',
                ],
                'invalid_message' => 'Les mots de passe doivent correspondre.',
            ]);
    }

    public function configureOptions(OptionsResolver $resolver): void
    {
        $resolver->setDefaults([
            'data_class' => User::class,
        ]);
    }
}
\end{lstlisting}

% ============================================================================
%                    CHAPITRE 9 : TEMPLATES TWIG
% ============================================================================
\chapter{Templates Twig - La Vue}

\section{Héritage de Templates}

Twig utilise un système d'\textbf{héritage} avec \code{extends} et \code{block}.

\begin{lstlisting}[style=twigstyle, caption={base.html.twig - Structure}]
<!DOCTYPE html>
<html lang="fr">
<head>
    <meta charset="UTF-8">
    <meta name="viewport" content="width=device-width, initial-scale=1.0">
    <title>Bienvenue - FreshMarket</title>
    
    
        <link href="https://cdn.jsdelivr.net/npm/bootstrap@5.3.0/dist/css/bootstrap.min.css" rel="stylesheet">
        <link href="{{ asset('styles/app.css') }}" rel="stylesheet">
    
</head>
<body>
    
    
    <main>
        
    </main>
    
    
    
    
        <script src="https://cdn.jsdelivr.net/npm/bootstrap@5.3.0/dist/js/bootstrap.bundle.min.js"></script>
    
</body>
</html>
\end{lstlisting}

\section{Variables et Filtres}

\begin{lstlisting}[style=twigstyle, caption={Affichage de données}]
{# Variables #}
{{ product.nom }}
{{ product.prix|number_format(2, ',', ' ') }} EUR

{# Conditions #}

    <span class="badge bg-success">En stock</span>

    <span class="badge bg-danger">Rupture</span>


{# Boucles #}

    <tr>
        <td>{{ item.product.nom }}</td>
        <td>{{ item.quantity }}</td>
        <td>{{ item.product.prix * item.quantity }} EUR</td>
    </tr>

    <tr><td colspan="3">Panier vide</td></tr>

\end{lstlisting}

% ============================================================================
%                    CHAPITRE 10 : DÉPLOIEMENT
% ============================================================================
\chapter{Déploiement et Configuration}

\section{Fichiers de Configuration}

\begin{table}[H]
\centering
\caption{Fichiers d'Environnement}
\begin{tabularx}{\textwidth}{|l|X|}
\hline
\textbf{Fichier} & \textbf{Usage} \\
\hline
\hline
\code{.env} & Variables par défaut (versionnées) \\
\hline
\code{.env.local} & Variables locales (non versionnées) \\
\hline
\code{.env.prod} & Variables de production \\
\hline
\code{.env.test} & Variables pour les tests \\
\hline
\end{tabularx}
\end{table}

\section{Docker}

\begin{lstlisting}[style=yamlstyle, caption={docker-compose.yml}]
version: '3.8'
services:
  web:
    build: .
    ports:
      - "8080:80"
    volumes:
      - .:/var/www/html
    depends_on:
      - database

  database:
    image: mariadb:10.11
    environment:
      MYSQL_ROOT_PASSWORD: secret
      MYSQL_DATABASE: freshmarket
    volumes:
      - db_data:/var/lib/mysql

volumes:
  db_data:
\end{lstlisting}

% ============================================================================
%                           CONCLUSION
% ============================================================================
\chapter{Conclusion}

Ce rapport technique a présenté en détail l'architecture et l'implémentation du projet \textbf{FreshMarket}, une application e-commerce complète développée avec Symfony 7.3. 

\section{Récapitulatif des Points Clés}

\begin{itemize}
    \item \textbf{Architecture MVC} : Séparation stricte entre Entités, Contrôleurs et Templates.
    \item \textbf{Doctrine ORM} : Mapping objet-relationnel avec validation intégrée.
    \item \textbf{Sécurité Multicouche} : OAuth, CSRF, Headers HTTP, Hashage bcrypt/argon2.
    \item \textbf{Services Réutilisables} : CartService encapsulant la logique métier.
    \item \textbf{Frontend Moderne} : Stimulus/Turbo pour une expérience SPA légère.
\end{itemize}

\section{Points d'Amélioration Suggérés}

\begin{enumerate}
    \item Implémenter la décrémentation automatique du stock lors d'une commande.
    \item Ajouter un système de paiement (Stripe, PayPal).
    \item Mettre en place des tests unitaires et fonctionnels avec PHPUnit.
    \item Configurer un pipeline CI/CD (GitHub Actions, GitLab CI).
\end{enumerate}

\appendix
\chapter{Glossaire}

\begin{description}
    \item[ORM] Object-Relational Mapping - Technique de correspondance entre objets et tables.
    \item[CSRF] Cross-Site Request Forgery - Attaque forçant un utilisateur à exécuter des actions.
    \item[XSS] Cross-Site Scripting - Injection de scripts malveillants.
    \item[OAuth] Protocole d'autorisation permettant l'authentification via des tiers.
    \item[Autowiring] Injection automatique de dépendances par Symfony.
\end{description}

\end{document}
 